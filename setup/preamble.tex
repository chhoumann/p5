\documentclass[11pt,twoside,a4paper,openany]{report}
%%%%%%%%%%%%%%%%%%%%%%%%%%%%%%%%%%%%%%%%%%%%%%%%
% Language, Encoding and Fonts
% http://en.wikibooks.org/wiki/LaTeX/Internationalization
%%%%%%%%%%%%%%%%%%%%%%%%%%%%%%%%%%%%%%%%%%%%%%%%
% Select encoding of your inputs. Depends on
% your operating system and its default input
% encoding. Typically, you should use
%   Linux  : utf8 (most modern Linux distributions)
%            latin1 
%   Windows: ansinew
%            latin1 (works in most cases)
%   Mac    : applemac
% Notice that you can manually change the input
% encoding of your files by selecting "save as"
% an select the desired input encoding. 
\usepackage[utf8]{inputenc}
% Make latex understand and use the typographic
% rules of the language used in the document.
\usepackage[english]{babel}
% Use the palatino font
\usepackage[sc]{mathpazo}
\linespread{1.05}         % Palatino needs more leading (space between lines)
% Choose the font encoding
\usepackage[T1]{fontenc}

%%%%%%%%%%%%%%%%%%%%%%%%%%%%%%%%%%%%%%%%%%%%%%%%
% Graphics and Tables
% http://en.wikibooks.org/wiki/LaTeX/Importing_Graphics
% http://en.wikibooks.org/wiki/LaTeX/Tables
% http://en.wikibooks.org/wiki/LaTeX/Colors
%%%%%%%%%%%%%%%%%%%%%%%%%%%%%%%%%%%%%%%%%%%%%%%%
% load a colour package
\usepackage{xcolor}
\definecolor{aaublue}{RGB}{33,26,82}% dark blue
% The standard graphics inclusion package
\usepackage{graphicx}
% Set up how figure and table captions are displayed
\usepackage{caption}
\captionsetup{%
  font=footnotesize,% set font size to footnotesize
  labelfont=bf % bold label (e.g., Figure 3.2) font
}
% Make the standard latex tables look so much better
\usepackage{array,booktabs}
% Enable the use of frames around, e.g., theorems
% The framed package is used in the example environment
\usepackage{framed}

% Adds support for full page background picture
\usepackage[contents={},color=gray]{background}
%\usepackage[contents=draft,color=gray]{background}

%%%%%%%%%%%%%%%%%%%%%%%%%%%%%%%%%%%%%%%%%%%%%%%%
% Mathematics
% http://en.wikibooks.org/wiki/LaTeX/Mathematics
%%%%%%%%%%%%%%%%%%%%%%%%%%%%%%%%%%%%%%%%%%%%%%%%
% Defines new environments such as equation,
% align and split 
\usepackage{amsmath}
% Adds new math symbols
\usepackage{amssymb}
% Use theorems in your document
% The ntheorem package is also used for the example environment
% When using thmmarks, amsmath must be an option as well. Otherwise \eqref doesn't work anymore.
\usepackage[framed,amsmath,thmmarks]{ntheorem}

%%%%%%%%%%%%%%%%%%%%%%%%%%%%%%%%%%%%%%%%%%%%%%%%
% Page Layout
% http://en.wikibooks.org/wiki/LaTeX/Page_Layout
%%%%%%%%%%%%%%%%%%%%%%%%%%%%%%%%%%%%%%%%%%%%%%%%
% Change margins, papersize, etc of the document
\usepackage[
  inner=28mm,% left margin on an odd page
  outer=41mm,% right margin on an odd page
  ]{geometry}
% Modify how \chapter, \section, etc. look
% The titlesec package is very configureable
\usepackage{titlesec}
\titleformat{\chapter}[display]{\normalfont\huge\bfseries}{\chaptertitlename\ \thechapter}{20pt}{\Huge}
\titleformat*{\section}{\normalfont\Large\bfseries}
\titleformat*{\subsection}{\normalfont\large\bfseries}
\titleformat*{\subsubsection}{\normalfont\normalsize\bfseries}
%\titleformat*{\paragraph}{\normalfont\normalsize\bfseries}
%\titleformat*{\subparagraph}{\normalfont\normalsize\bfseries}

% Clear empty pages between chapters
\let\origdoublepage\cleardoublepage
\newcommand{\clearemptydoublepage}{%
  \clearpage
  {\pagestyle{empty}\origdoublepage}%
}
\let\cleardoublepage\clearemptydoublepage

% Change the headers and footers
\usepackage{fancyhdr}
\pagestyle{fancy}
\fancyhf{} %delete everything
\renewcommand{\headrulewidth}{0pt} %remove the horizontal line in the header
\fancyhead[RE]{\small\nouppercase\leftmark} %even page - chapter title
\fancyhead[LO]{\small\nouppercase\rightmark} %uneven page - section title
\fancyhead[LE,RO]{\thepage} %page number on all pages
% Do not stretch the content of a page. Instead,
% insert white space at the bottom of the page
\raggedbottom
% Enable arithmetics with length. Useful when
% typesetting the layout.
\usepackage{calc}

%%%%%%%%%%%%%%%%%%%%%%%%%%%%%%%%%%%%%%%%%%%%%%%%
% Bibliography
% http://en.wikibooks.org/wiki/LaTeX/Bibliography_Management
%%%%%%%%%%%%%%%%%%%%%%%%%%%%%%%%%%%%%%%%%%%%%%%%
\usepackage[backend=bibtex,
  bibencoding=utf8,
  style=numeric-comp
  ]{biblatex}
\addbibresource{bib/mybib}

%%%%%%%%%%%%%%%%%%%%%%%%%%%%%%%%%%%%%%%%%%%%%%%%
% Misc
%%%%%%%%%%%%%%%%%%%%%%%%%%%%%%%%%%%%%%%%%%%%%%%%
% Hide the ugly red borders around clickable hyperlinks/references
\usepackage[hidelinks]{hyperref}
% Add bibliography and index to the table of
% contents
\usepackage[nottoc]{tocbibind}
% Add the command \pageref{LastPage} which refers to the
% page number of the last page
\usepackage{lastpage}
% Add todo notes in the margin of the document
\usepackage[
%  disable, %turn off todonotes
  colorinlistoftodos, %enable a coloured square in the list of todos
  textwidth=\marginparwidth, %set the width of the todonotes
  textsize=scriptsize, %size of the text in the todonotes
  ]{todonotes}

%%%%%%%%%%%%%%%%%%%%%%%%%%%%%%%%%%%%%%%%%%%%%%%%
% Hyperlinks
% http://en.wikibooks.org/wiki/LaTeX/Hyperlinks
%%%%%%%%%%%%%%%%%%%%%%%%%%%%%%%%%%%%%%%%%%%%%%%%
% Enable hyperlinks and insert info into the pdf
% file. Hypperref should be loaded as one of the 
% last packages
\usepackage{hyperref}
\hypersetup{%
	pdfpagelabels=true,%
	plainpages=false,%
	pdfauthor={Author(s)},%
	pdftitle={Title},%
	pdfsubject={Subject},%
	bookmarksnumbered=true,%
	colorlinks=false,%
	citecolor=black,%
	filecolor=black,%
	linkcolor=black,% you should probably change this to black before printing
	urlcolor=black,%
	pdfstartview=FitH%
}

%%%%%%%%%%%%%%%%%%%%%%%%%%%%%%%%%%%%%%%%%%%%%%%%
% Listings (Code snippets)
% https://en.wikibooks.org/wiki/LaTeX/Source_Code_Listings
%%%%%%%%%%%%%%%%%%%%%%%%%%%%%%%%%%%%%%%%%%%%%%%%
\usepackage{listings}
\usepackage{color}

\renewcommand{\lstlistingname}{Code snippet} % Listing -> Code snippet
\definecolor{lighter-gray}{RGB}{240,240,240}

\lstset{
  backgroundcolor=\color{lighter-gray},
  extendedchars=true,
  basicstyle=\footnotesize\ttfamily,
  showstringspaces=false,
  showspaces=false,
  numbers=left,
  tabsize=4,
  breaklines=true,
  showtabs=false,
  captionpos=b,
  numberstyle=\footnotesize,
  numbersep=5pt
}

\usepackage{float}

% Define C# as a snippet language.
\usepackage{courier}

\definecolor{Green}{rgb}{0, 0.3, 0}
\definecolor{DarkCyan}{rgb}{0, 0.545, 0.545}
\definecolor{Navy}{rgb}{0, 0, 0.5}
\definecolor{Teal}{rgb}{0, 0.5, 0.5}
\definecolor{DarkGray}{gray}{0.66}
\definecolor{Olive}{rgb}{0.5, 0.5, 0}
\definecolor{Pink}{rgb}{1.0, 0.75, 0.8}
\definecolor{DeepPink}{rgb}{1, 0.08, 0.58}
\definecolor{Brown}{rgb}{0.65, 0.165, 0.165}
\definecolor{DarkViolet}{rgb}{0.58, 0, 0.83}
\definecolor{SaddleBrown}{rgb}{0.55, 0.27, 0.07}
\definecolor{Orange}{rgb}{0.9, 0.427, 0}
\lstdefinelanguage{CSharp}{
  morecomment = [l]{//}, 
  morecomment = [l]{///},
  morecomment = [s]{/*}{*/},
  morestring=[b]", 
  morestring=[b]',
  basicstyle=\footnotesize\ttfamily,
  commentstyle=\color{Green}\textit,
  stringstyle=\color{Orange},
  sensitive = true,
  morekeywords=[1]{this, base},
  keywordstyle=[1]\bfseries\color{Navy},
  morekeywords=[2]{as, is, new, sizeof, typeof, true, false, stackalloc},
  keywordstyle=[2]\color{Navy}\bfseries,
  morekeywords=[3]{else, if, switch, case, default,
  do, for, foreach, while, in},
  keywordstyle=[3]\bfseries\color{Navy} ,
  morekeywords=[4]{break, continue, goto, return,
  yield, partial, global, where},
  keywordstyle=[4]\bfseries\color{Navy},
  morekeywords=[5]{try, throw, catch, finally},
  keywordstyle=[5]\color{Navy}\bfseries,
  morekeywords=[6]{checked, unchecked},
  keywordstyle=[6]\color{Navy}\bfseries,
  morekeywords=[7]{fixed, unsafe},
  keywordstyle=[7]\bfseries\color{Navy},
  morekeywords=[8]{bool, byte, sbyte, char, short, ushort, int, uint, long, ulong, float,
  double, decimal, enum, struct},
  keywordstyle=[8]\bfseries\color{Navy},
  morekeywords=[9]{class, interface, delegate, object, string,
  void},
  keywordstyle=[9]\bfseries\color{Navy},
  morekeywords=[10]{explicit, implicit, operator},
  keywordstyle=[10]\bfseries\color{Navy},
  morekeywords=[11]{params, ref, out},
  keywordstyle=[11]\bfseries\color{Navy},
  morekeywords=[12]{private, protected, internal, public},
  keywordstyle=[12]\bfseries\color{Navy},
  morekeywords=[13]{abstract, const, event, var, override, virtual, volatile, extern, readonly, sealed, static},
  keywordstyle=[13]\bfseries\color{Navy},
  morekeywords=[14]{namespace, using},
  keywordstyle=[14]\bfseries\color{Navy},
  morekeywords=[15]{lock},
  keywordstyle=[15]\bfseries\color{Navy},
  morekeywords=[16]{get, set, add, remove},
  keywordstyle=[16]\bfseries\color{Navy},
  morekeywords=[17]{null, value},
  keywordstyle=[17]\bfseries\color{Navy},
}
\newcommand{\userStory}[3]{
  \begin{center}
    \begin{tabular}{|p{0.8\textwidth}|} 
     \hline
     \textbf{USER STORY}\\
     As \textit{#1}\\
     I want #2\\
     so I #3.\\ [0.5ex] 
     \hline
    \end{tabular}
    \end{center}  
    }
\usepackage{multirow}

\titleformat*{\subsubsection}{\large\bfseries}
\titleformat*{\subsection}{\Large\bfseries}
\titleformat*{\section}{\LARGE\bfseries}