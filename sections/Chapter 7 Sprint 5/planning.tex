\section{Sprint Planning}
%The sprint goal for this group
%How does our sprint goal help with the overall sprint goal
%The purpose and priority for each increment
%Show the available backlog

\subsection{User stories}
The \knox{} product owner have given us three user stories, which we can use to detail the coming tasks for the sprint.
Our own product owner will split these user stories, which we consider to be \textit{epics}, together with the involving layer \cite{Epics}.
This will help us understand how to accommodate the needs of the involving layer and prioritize the stories.
This in turn will help us generate as much value to the pipeline as quickly as possible \cite{UserStories}.

\userStory{the knowledge layer}{insert Triples into a storage}{others can access it later}
\userStory{the functionality layer}{query an RDF graph}{can use it in search results, fact checking and virtual assistant applications}
\userStory{the functionality layer}{query an RDF graph in a standardized way}{do not have to worry about changes to the database}

\subsubsection*{Granulating the epics}
After talking to the functionality layer, we found out some of the details about how they want database access to work. 

\begin{enumerate}
    \item It should be possible to write RDF data to the Fuseki database using the Terse RDF Triple Language (Turtle) syntax and file format.
    \item They want to be able to send SPARQL queries directly to an endpoint to fetch the data.
    \item They would prefer if the response was in XML format. They have given us an example.
    \item They do not know the size of the data they will be querying.
    \item The route for GET request should have a string parameter called "query" containing the SPARQL query as a string.
    The query will be in URL-coded format.
\end{enumerate}

We have gotten enough insight to begin work on the database, however due to the other groups not knowing exactly what they want, the planning for this sprint will be a little ad hoc and unpredictable.  
This means that there is a possibility of newly discovered works during the sprint.

With this in mind, we first transferred the remaining report backlog from the last sprint and recieved three user stories from the PO. 
From this backlog, we have formulated the goals for this sprint.

From the epics, we created a backlog. As mentioned in section \ref{Whatwelearnedsprint3}, we decided to work with a different unit of mesurement for determining the size of a task.
Instead of using hours, we now simply interpret the points assigned to a task as what determines its size.

We partook in a game of Inversion \todo{citation needed} where we tried to find problems with our ideas by attempting to see them from the point of view of a potential examinator. 
We took the items from the backlog that we defined as fulfilling our sprint goal, and tried to think about what would happen if we failed to fulfill them. 
This was done to validate the tasks — making sure that they are absolutely necessary. 
This was only done to the implementation tasks as they have the most significant impact on \knox{} as a whole.
