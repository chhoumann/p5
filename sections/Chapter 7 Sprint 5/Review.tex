\section{Review}
This section will review the accomplishment for the 5th sprint as well as changes in the \knox{} environment. 

The goal for this sprint was to implement functionality to support the user stories defined in section \ref{sec:userstories5}. 

\subsubsection*{What Had Been Accomplished?}
The user stories which referred to the RDF database were implemented but were all yet to be deployed to the \knox{} pipeline. This meant that the acceptance criteria proposed in section \ref{acceptCriteriaSprint5} was fulfilled, but since the features were not deployed to the pipeline, the primary goal for the sprint was not accomplished. 

\subsubsection*{What Had Changed in the Environment?}
During the sprint, we discovered that many better solutions for RDF data storage exist. Therefore, the group chose to not implement an endpoint to the Fuseki service, but implement Virtuoso instead. 

\subsubsection*{What Next?}
The \knox{} PO informed us that he would not give any of the \knox{} groups more assignments regarding the pipeline. 
For the next sprint, we planned on continuing with the remaining user stories as well as deploying the endpoint for the database.
However, each of the \knox{} groups had to implement a UI element. We prioritized doing this in the coming sprint.
We also wanted to restructure the database design in the WordCount database, as discussed in section \ref{InitialDesign}.


In summary, we would like to solve the following problems next sprint.

\begin{itemize}
    \item Create a simple UI element for the \knox{} website.
    \item Redesign the database schema for WordCount.
    \item Deploy the API endpoint for the RDF database.
    \item Record a video that shows the purpose of the individual components of the database layer.
\end{itemize}
