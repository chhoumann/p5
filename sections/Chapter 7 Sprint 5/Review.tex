\section{Review}
This section will review the accomplishment for the 5th sprint as well as changes in the \knox{} environment. 

The goal for this sprint was to implement functionality to support the user stories defined in section \ref{sec:userstories5}. 

\subsubsection*{What has been accomplished?}
The user stories which refer to the RDF database have implemented but are all yet to be deployed to the \knox{} pipeline. This means the acceptance criteria proposed in section \ref{acceptCriteriaSprint5} is fulfilled, but since the features were not deployed to the pipeline, the primary goal for the sprint has not been accomplished. 

\subsubsection*{What has changed in the environment?}
During the sprint, we discovered that many better solutions for RDF data storage exist. Therefore, the group chose to not implement an endpoint to the Fuseki service, but implement Virtuoso instead. 

\subsubsection*{What next?}
The PO has informed us that he will not give any of the \knox{} groups more assignments regarding the pipeline. 
For the next sprint, we will continue on the remaining user stories and deploy the endpoint for the database.
However, each of the \knox{} groups must implement a UI element. We will prioritize doing this as well.
We would also like to restructure the database design in the WordCount database, as discussed in section \ref{InitialDesign}.


In summary, we would like to solve the following problems next sprint.

\begin{itemize}
    \item Create a simple UI element for the \knox{} website.
    \item Redesign the database schema for WordCount.
    \item Deploy the API endpoint for the RDF database.
    \item Record a video that shows the purpose of the individual components of the database layer.
\end{itemize}
