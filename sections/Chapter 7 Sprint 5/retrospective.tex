\section{Retrospective}
We have received three user stories from the PO, but have not deployed any of them to the \knox pipeline. Thus, we have not completed the sprint goal.

\subsection{Sprint goal}
The goal of this sprint was to provide endpoints for an RDF database, such that the \knox pipeline could fetch data using a SPARQL query, and input data in turtle format.
Since none of these features are deployed, we do not meet the acceptance criteria (see section \ref{acceptCriteriaSprint5}).



\subsection{Reflection on the sprint}
During the sprint, we were made aware that the report needed more theory on databases. 
Therefore some of the tasks regarding report writing grew to an almost unmanageable size. This in turn meant that more time was spend on writing.
The shift from Fuseki to Virtuoso did not make a big impact on the project, but some time was wasted on researching how to connect to the Fuseki service. 

During previous sprints, the PO had not given us user stories (epics). Having user stories greatly improved our focus when grooming the backlog, and also helped us establish when a tasks was done. 
Granulating the user stories with the groups from the layers involved in the stories helped us eliminate a task; 
originally the PO did not want endpoints taking SPARQL requests as input.
The knowledge layer disagreed, and after talking to the PO, it was established that a SPARQL endpoint would do just fine.


\subsection{What we learned}
