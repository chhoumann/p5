\section{Current state}

As previously mentioned, \knox{} was initially started in 2020.
The previous group chose to develop this layer using Java.

Using the framework Apache Jena in combination with the Fuseki package, they set up a simple database system.

The system has two functionalities. 
Firstly, it can store knowledge graphs in HDT and convert triples to HDT.
Secondly, it contains a prototype for counting the number of words in an article.
There is no API for reading this data, only for writing. 
In addition, the other layers are directly connected to the database.

Overall, not much code was written and its quality is questionable. 
The Java language convention was not followed, no documentation was made, and only a single test exists. 
It appears as though no structure was established, making it difficult to navigate the code. 
Furthermore, the database is set up such that is must be restarted every time new data is written to it - otherwise, the data cannot be fetched.

Based on this, we have discussed what work to do and in what order.
After having read and understood the current code, we will have to start by cleaning up the inconsistencies and structure.
We will then have to document what the code does and write tests for it.
We must also address the unfortunate database implemenation such that constantly restarting is no longer required.
Finally, we need to decouple the other layers such that they are no longer directly connected to the database.

To solve these issues, it was decided that the best approach was to discard all current implementations. 
This decision was based on several factors such as unfamilarity with the environment and the previously mentioned poor structure. 
Discarding the current implementation also allows us to build the layer using C#, a programming language that we are more familiar with which is also taught in the future semesters. 
This will make it easier for the following groups to continue and not have to learn a new language.

While doing so, we will follow a proper structure and write tests and documentation along the way. 
Moreover, it would make the system more accessible to future students as C# is taught on the 3rd semester.  
