\chapter{Sprint 3}
We have chosen to combine our descriptions of the two previous sprints into one sprint. Sprint 2 was supposed to be the sprint where the search engine was fixed, but ended up being swallowed by planning and bureaucracy. For this reason, it also took sprint 3 to fully implement the search engine. Because of our choice to combine the two sprints, any reference to sprint 3 should be interpreted as referring to the combined sprint.

%Common / Overall goal for all Knox groups
\chapter{Introduction to the search engine}
For the third sprint, the various groups of the Knox project were tasked with finishing the implementation of the Search Engine. The goal of the Search Engine is to search through files provided by \textit{Grundfos} and \textit{Nordjyske Mediehus}, using phrases or words. When the groups received the project, the Search Engine was non-functional. Issues in the different components resulted in the groups collaborating on fixing current modules or implementing new ones. Many of the problems inherited from last years' implementations were results of miscommunications between the pipeline layers. These issues mainly occurred due to inconsistency in the structure of objects sent between the different layers. Other issues included a lack of documentation and routing issues. 

The preprocessing layer experienced problems with parsing some files, as well as API communication with the knowledge layer. The knowledge layer did not have an established connection to the Data layer and could not  add the entries to the database. The data layer did not have working APIs for accessing the database and the functionality layer, therefore, accessed the wordcount database directly. At this point in time, the database only contained test data.

To settle these issues, first a product backlog, and after some time an MVP was defined for the search engine. The latter being defined as: “All non-problematic articles from \textit{Grundfos} and \textit{Nordjyske Mediehus} from 2017-2021 must be queryable from the UI in natural language, with an option for choosing which source to search in, and with a link to the original article.” 
The backlog contained tasks for each of the Knox project groups.
\section{Sprint Planning}
As noted in the preceding sprint, the codebase inherited from the previous iteration was deemed problematic and costly to maintain. 
Therefore our sprint goal for this iteration was based on updating the codebase to better comply with the requirements for the \knox{} goal.


Based on this observation as well as the aforementioned requirements, our sprint goal was as follows:
\vspace{\baselineskip}

\textit{Make the minimally required amount of API endpoints available to satisfy the requirements for the search engine, and create the necessary documentation.}

\subsubsection{Backlog \& Increments}
After defining the sprint goal, it was granulated into subtasks. 
With the tasks granulated, a session of Scrum poker was held in order to evaluate the time requirement of each task. 
Afterwards, the tasks were categorized by their priority and focus points.
Lastly, the tasks were divided into three different sprint increments: \textit{Wordcount API}, \textit{Write Report}, \textit{Wiki Documentation}. 


%%Remember to add an estimate to Write report and wiki update
\begin{table}[]
\begin{tabular}{|l|l|l|ll}
\cline{1-3}
Estimate & Increments                         & Tasks                         &  &  \\ \cline{1-3}
82       & \multicolumn{1}{c|}{Wordcount API} & CRUD operations on controller &  &  \\ \cline{1-3}
X        & Write Report                       & Document the process          &  &  \\ \cline{1-3}
X        & Wiki Update                        & Feature documentation         &  &  \\ \cline{1-3}
\end{tabular}
\end{table}





%\subsection{Sprint goal}
%The sprint goal for this group
%How does our sprint goal help with the overall sprint goal
%The purpose and priority for each increment
%Show the available backlog
\todo{sprint goal - as defined in collaboration with other groups and PO (Section \ref{})}
This chapter covers the development process during the third sprint. The third sprint covers the implementation of the needed database features for getting a functioning search engine. 


To accomplish this goal, the product owner specified a list of expectations for the entire \knox{} project that, when completed, would fulfill a minimal viable product. The specified list of common goals is outlined below.
\begin{itemize}
	\item It should be possible to make searches in all the available databases.
	\item It should be possible to make searches using keywords in all the available data.
	\item It should be possible to make searches in all Nordjyske articles from 2017-2021
	\item It should be possible to make searches in all the Grundfos manuals, barring those that have shown to be problematic during processing.
	\item The UI should at least be able to display a link to the given article but preferably also display the title and text.
\end{itemize}

With the overall goal for the sprint outlined, a granulated sprint goal could be made that accounts for the focus of our scrum group.

\section{Increment}
With the aforementioned tasks in mind, we will now move on to describing the increment of this sprint.

We start by outlining the different database frameworks considered for the project.
Afterwards, we discuss which framework was chosen and why.

\subsection{Models}
% to implement our api we used these models

% Communication models

    % JSON input models
        % ArticleJsonModel
        % JsonSchemaModel
        % TermJsonModel

    % response models
        % FileIdResponse


% to acces the DB using EF core we used these models
% data access models
    % Article
    % JsonSchemaModel
    % pubisher
    % Term
    % WordRatio

\subsection{CRUD}

As mentioned, the main focus of this sprint was to get the server up and running for the other layers. 
As we already had a working database, the goal of this increment was to write a CRUD API in C\# such that the other layers could use to access the database via HTTP requests.
The CRUD operations we implemented are based on requests from the other layers, and what they thought they needed to retrieve from the database.

To implement this, we split the CRUD operations into three different APIs handling different aspects of the database. 
The pipeline for these APIs is outlined in figure \ref{Node02Sprint3}.

\begin{figure}[h]
    \centering
    \includegraphics[width=\linewidth]{Images/Node02Sprint3.PNG}
    \caption{Node02 server pipeline in sprint 3.}
    \label{Node02Sprint3}
\end{figure}

\subsubsection{WordCount API}
The first API we created was the WordCount API which has the route \texttt{/wordcount}. Here we implemented a \texttt{GET} and a \texttt{POST} method.
The \texttt{GET} method, simply called \texttt{GetFilepath}, takes a integer id as an argument and finds an article with the given id and returns the article.
In the WordCount API, we also implemented a \texttt{POST} method called \texttt{Post} which takes a JSON object from the HTTP body as a parameter. 
The JSON object should correspond to an array of articles which will then be posted to the database. 

Before posting the data to the database, the method checks whether the article title, file path or words already exists in the database, and whether the input matches the JSON schema from the database.

\begin{table}[]
\begin{tabular}{|llll|}
\hline
\multicolumn{4}{|c|}{\textbf{WordCount API}}                                                                                 \\ \hline
\multicolumn{1}{|l|}{Name}                 & \multicolumn{1}{l|}{Method} & \multicolumn{1}{l|}{Input}       & Response on success       \\ \hline
\multicolumn{1}{|l|}{\texttt{GetFilepath}} & \multicolumn{1}{l|}{GET}    & \multicolumn{1}{l|}{Integer}     & Article        \\ \hline
\multicolumn{1}{|l|}{\texttt{Post}}        & \multicolumn{1}{l|}{POST}   & \multicolumn{1}{l|}{JSON Object} & Status message \\ \hline
\end{tabular}
\end{table}


\subsubsection{WordRatios API}
The WordRatios API is a view used by layer 4 to fetch specific data from the database.
As it is a view, it cannot be used to store new data. 
The primary purpose of the WordRatios API is to check how many times a given word occours in an article.
The route to this endpoint is \texttt{/wordratio} and the API consists of two \texttt{GET} methods. 
The first method \texttt{GetAllWordRatios} is used to get all the entries in the WordRatios table.
The second method \texttt{GetMatches} is used to query one or more specific words. 
One may provide one or more sources in which to search for the specified words.
If no sources are provided, the database searches for the given words in all sources.

\begin{table}[]
\begin{tabular}{|llll|}
\hline
\multicolumn{4}{|c|}{\textbf{WordRatios API}}                                                                                                                                           \\ \hline
\multicolumn{1}{|l|}{Name}                      & \multicolumn{1}{l|}{Method} & \multicolumn{1}{l|}{Input}                           & Reponse on success                               \\ \hline
\multicolumn{1}{|l|}{\texttt{GetAllWordRatios}} & \multicolumn{1}{l|}{GET}    & \multicolumn{1}{l|}{None}                            & WordRatio Table                                  \\ \hline
\multicolumn{1}{|l|}{\texttt{GetMatches}}       & \multicolumn{1}{l|}{GET}    & \multicolumn{1}{l|}{term (string), sources (string)} & WordRatio for all articles with term from source \\ \hline
\multicolumn{1}{|l|}{\texttt{GetMatches}}       & \multicolumn{1}{l|}{GET}    & \multicolumn{1}{l|}{term (string)}                   & WordRatio for all articles with term             \\ \hline
\end{tabular}
\end{table}

\subsubsection{Schema API}

The last API we implemented in this iteration is the Schema API which is used to post and retrieve a JSON schema from the database. These are later used to verify JSON objects.
The route to this endpoint is simply \texttt{/Schema}. 
This API concists of two \texttt{GET} methods and a single \texttt{POST} method. The first \texttt{GET} method is called GetSchema which takes a SchemaName as its input and returns a JSON Schema. The second \texttt{GET} method is called GetAllSchemas which takes no input and returns all stored JSON schemas. The last method is a \texttt{POST} method called PostJSONSchema and takes a JSON Schema as its input and inserts it into the database.  
The body of the post request must consist of two fields - the schema name, which is its primary key in the database, and the schema content itself. 
A schema gets stored in the database in the format known as \texttt{jsonb} or JSON binary which is a built-in type provided by \postgres{}.
This format is used to store a JSON object as binary which uses much less space than a varchar or regular JSON type would.

\begin{table}[]
\begin{tabular}{|llll|}
\hline
\multicolumn{4}{|c|}{\textbf{Schema API}}                                                                                                     \\ \hline
\multicolumn{1}{|l|}{Name}                     & \multicolumn{1}{l|}{Method} & \multicolumn{1}{l|}{Input}               & Response on success \\ \hline
\multicolumn{1}{|l|}{\textit{GetSchema}}       & \multicolumn{1}{l|}{GET}    & \multicolumn{1}{l|}{SchemaName (string)} & JSON Schema         \\ \hline
\multicolumn{1}{|l|}{\textit{GetAllSchemas}}   & \multicolumn{1}{l|}{GET}    & \multicolumn{1}{l|}{None}                & All JSON Schemas    \\ \hline
\multicolumn{1}{|l|}{\textit{PostJSONSchemas}} & \multicolumn{1}{l|}{POST}   & \multicolumn{1}{l|}{JSON Schema}         & Status message      \\ \hline
\end{tabular}
\end{table}

\todo{Mention earlier that the database was accesed directly, instead of through an API}
\todo{Describe the structure of the database}

We chose not to implement the update and delete operations in this increment as it was not strictly needed to make the search engine work.
However, we have discussed doing so in a future increment.
\section{Validation}

As previously mentioned in \ref{}, the knowledge layer posts data to the database that the functionality need to use. 
In order to ensure that the data being posted is correct, we have developed a validation method in collaboration with the knowledge layer.

The knowledge layer posts their data in the JSON format. 
To validate JSON files, one must develop a JSON schema that specifies a structure that a JSON file can be checked against.

We created a new controller \texttt{SchemaController} which can be used to post JSON schemas to the database. 
Currently, only one schema is stored in the database, however the system is developed such that it is scalable and easily expandable.

In the WordCount controller, the \texttt{POST} method first fetches the JSON schema from the database.
A new object of type \texttt{JsonValidator} is then instantiated, and its \texttt{IsValid} method is then called to determine whether the input conforms to the structure defined within the given schema. 
The method returns true if the JSON object follows the schema, and false if it does not
The class is generic, allowing for reusability for multiple classes, and it is implemented using the Newtonsoft JSON framework. 

This can be seen in code snippet \ref{lst:json_post}.

\begin{lstlisting}[language=CSharp, caption={Snippet from the \texttt{POST} method showing validation of the input JSON.}, label={lst:json_post}]
[HttpPost]
public IActionResult Post([FromBody] JsonElement jsonElement)
{
	JsonSchemaModel? schema = unitOfWork.SchemaRepository.Find(s => s.SchemaName == WordCountSchemaName);
	
	\dots

	// Get schema and use for validating
	if (!new JsonValidator<ArticleJsonModel[]>(schema.JsonString)
		.IsValid(jsonInput, out ArticleJsonModel[] jsonArticles))
	{
		return BadRequest("Wrong body syntax, does not follow schema.");
	}
}
\end{lstlisting}

The \texttt{IsValid} method first checks whether the provided JSON string matches the provided JSON schema.
If so, it simply deserializes the JSON string into the class provided by the generic constraint.
This is captured in code snippet \ref{jsonIsValid}.

\begin{lstlisting}[language=CSharp, caption={The \texttt{IsValid} method from the \texttt{JsonValidator} class.}, label={lst:jsonIsValid}]
public bool IsValid(string jsonString, out T data)
{
	JToken jToken = JToken.Parse(jsonString);

	data = null;

	if (!jToken.IsValid(schema))
	{
		return false;
	}

	data = DeserializeJsonString(jsonString);

	return true;
}
\end{lstlisting}

If the result is valid, the deserialized object is returned as an \texttt{out} parameter for use by the caller.

\section{Database redesign}

As previously mentioned in section \ref{currentState}, the code from last year was left in an unfinished and undocumented state. We estimated that it would be faster and easier to scrap it and start over rather than spending time trying to understand and document it.
However, the database design seemed good enough to work with despite its flaws. 
We therefore decided to keep it as we were under time constraints.

In order to communicate with the database from the C\# application, we opted to use EF Core.
Because EF Core is designed with the Code First approach in mind, we are unable to simply make SQL queries in the code itself as the old group did. 
Instead, we had to model the current database design using C\# classes. 

This caused problems as the current design did not translate well into an object-oriented approach.
To execute what would otherwise be considered a relative simple SQL query, we had to write a ton of code which ultimately defeated the purpose of the Code First approach. 

It proved difficult to work with and caused more problems than benefits, especially under the aforementioned time constraints. 
Therefore, we decided to rework the design completely by creating C\# models that would translate to a relational database design using EF Core migrations.

The end result is much easier to work with and removed a lot of unnecessary complexity in the code.
However, despite the many advantages from the redesign, it has a major memory issue.
In our rush to quickly implement the redesign, we unfortunately overlooked how to effectively store words from articles. 
Currently, we just store every single word in the database without regard for duplicates, which takes up way more space than necessary.
Ideally, we should instead store each word only once with an associated counter for how many times the given appears in an article.
We realize the issues with the current design and would like to address them in the near-future if time permits it.
\subsection{Containerizing the API and database}
The final major goal of this sprint was to achieve the encapsulation that we described in the second sprint in section \ref{sec:Containerizing}. 
In order to achieve this, we had to add both the API and database to a container and create a network bridge between the two containers so that they could communicate with each other through a specified port.
\subsection{Continuous Integration and Delivery}
In our work, we have used both Continuous Integration (CI) and Continuous Delivery (CD).
For CI, we have used GitHub with GitHub Actions. We have created a GitHub Actions workflow that
runs the tests in our project; and another one that tests the code quality. These checks run
on every time commits are made towards the main branch. This ensures that we keep that branch production-ready
at all times.

For CD, we have implemented an automatic deployment system that deploys the code to the production server.
We start the workflow by tagging the commit with the version number. Once we push the tag, a GitHub Actions
workflow is triggered that deploys the code to the production server.
To facilitate the deployment, we containerized our software. The Action workflow starts by building the
Docker image, which is pushed to the GitHub Package Repository. On the production server, we have a Docker 
container watching for new releases. When a new release is pushed, the container pulls the new image and
deploys it.


\section{Review}



# Review

**What has been accomplished?**

- Get & post API's for the database (WordCount)
- Wordratio response objects
- Implemented CI/CD using docker and GitHub actions
- Restructured database using a code-first approach
- Repository-unit of work pattern has been implemented (encapsulation and separation of concerns)
- Development and production environment has been set up
- Report writing
- Wiki restructuring
- API for JSON schemas used for validating data sent to the API

**What has changed in the environment?**

- The structure of the objects' send from the below layer have changed; a 'publication' field has been introduced in the schema defining the object.
- We found out that the PO will not be creating a new backlog with tasks for us to solve, but that we should rather create meaningful tasks ourselves.
- We have made other groups realize that creating server-users, database schemas and other meaningless task are their own responsibility.

**What next?**

- During the scrum of scrums retrospective, we have pointed out, that we cannot keep jumping from task to task. Therefore, we have decided to implement a ticket system, to provide a structured approach to assisting other teams.
- At the scrum of scrum retrospect, many groups told that they want to work with knowledge graphs. Therefore, we should prepare for tasks regarding the RDF database.
- We should be prepared to provide assistance with the wordcount database endpoints. This included minor changes and new endpoint implementation.
-
\section{Retrospective}


\subsection{Sprint goal}
The goal of this sprint can be seen in \ref{sprint3Goal}.
All of the features planned during the sprint planning was implemented by the end of the sprint, but we did not document the code as well as we had initially planned. This clashes with our Definition Of Done, which means we did not fully reach the goal of this sprint.

\todo{ref to definition of done}

- **What was the goal of the sprint? Did we accomplish it?**
    
    We implemented all we wanted to implement, but did not document it as much as we had set out to. So no, we did not accomplish it.


    
- **What had been planned?**
    
    We planned out the implementation to a certain degree, but used a non-optimal format, which led to confusion down the line. We also did not use user-stories, which contributed to the confusion, as they would have otherwise provided a certain context for the tasks.
    
    The remaining tasks were lackluster / nonexistent, which led to trouble doing structured writing, as we had to figure out *what* to write every time we wanted to write.
    
- **What was done and what not? Plan vs. reality**
    
    Already answered in the goal bullet point. A reason for this was excessive confusion as the sprint progressed. To some extent, this was caused by conflicting information given by various sources. We also received a lot of 'urgent, not important' tasks from other groups, which we had to react to immediately.
    
    Another cause was our planning, which resulted in the aforementioned confusion.
    
    Not to mention that, due to all of the above, there were points where we did not exactly know what we had finished, what others in our groups were working on, and so on. Tasks were added almost daily, continuously changing the scope of the sprint.
    
- **Which training, skill, or knowledge contributed to the difference?** This documents our usage of the course syllabi subjects.
    
    We learned that we had to be more vigilant about narrowing the scope of our tasks. This was especially true when working on the API's where we initially thought that we should develop full CRUD capabilities, but it turned out that only some of the capabilities were necessary. From this we gathered that we should use some agile method to better our understanding of the needs of different actors/layers. Here we have chosen to use user stories to accommodate this. 
    
    Group members with previous experience using databases and EF Core also helped us develop the program much faster. 
    
- **What was added and what was removed from the sprint?**
    
    Literally too much was changed (added or removed) to say here. Lots of minor tasks were added/changed/removed constantly.
    
- **Did we hit our point estimation? If not, why?**
    
    Not at all. In some way, we went way off. This was due to a set of reasons including: Adding tasks that were not originally in the sprint, removing tasks that were and there were frequent interruptions from other groups which slowed progress on tasks.
    
- **What risks and problems were discovered? How did we solve them?**
    
    We lack domain knowledge about the data, data structure and general issues from the layers connecting to the database. We are operating in the dark, being blind from lack of knowledge. This means, that whenever we gain new insight in the domain, we have to ***QUICKLY** re-implement to accommodate.* 
    
- **Feedback from informants — if any.**
    
    We should not change response object structure - we could solve this by implementing classes that represent the response objects, instead of sending domain models. 
    
    Communication works! The other groups agreed that communication in-between groups (sitting in the same room) when testing works!
    
    Response from supervisor regarding report structure and content. This mainly regards itself with what is lacking/missing.
    
- **How did the iteration go wrong?**
    
    Lack of correct communication between groups, too much irrelevant communication. Contradicting information and tasks. Lack of production due to meetings that, by large, were not particularly useful. 
    
    Insufficient / poor planning.
    
    Too many distractions.
    
    Too much multi-tasking (which is horrible).
    
    Lack of internal communications (also slightly caused by not all members being present at all times) which has led to people not being caught up to all different aspects of our work.
    
- **What did we do well?**
    
    At some point, we implemented a bare minimum skeleton to provide content to other layers quickly. This made it possible for us to find issues quickly.
    
- **Which techniques were useful?**
    
    Using Scrum techniques to alleviate some of the issues. Pair programming(also writing). Daily Scrum, planning poker (what is our weaknesses in the group), kanban boards (although we are bad at updating).
    
- **Which techniques were not useful?**
    
    all techniques were useful, however we should be aware that we are not using them correctly. 
    
- **How can we improve in the next iteration? Make a plan for how we'll address the issues.**
    
    **Start doing**
    
    - User stories (product owner) from stakeholders
    - Specification of the data in the database - this should be given to the other groups so less issues regarding CRUD operations occur.
    - More schemas for communication between layers
    - Updating Kanban board
    - Change planning poker strategy:
        - if everyone is within two cards of each other, just average the sum of your cards.
        - If people are more than three cards apart, the high and low cards talk about why they think what they do.
        - don't estimate in hours, estimate in 'work' (size) - Scrum book has something to say about this
    - @Christian Bager Bach Houmann Velocity analysis
    - Component diagram versioning - we want to show the progress!
    - Tasks are **NOT** done before they are *released*. Will make a lane on the Kanban board as a landing for them while they wait for deployment.
    - Write testable code and test more. We will aim for 70% test coverage.
    - Sprints changed to 2-week duration. 4 is too long.
    
    **Continue doing**
    
    - Schemas, writing immediately after implementation
    - Development environments
    - Production deployments
    
    **Stop doing**
    
    - solving small tasks other groups introduce immediately.
- **Information about the next sprint. What should we do next?**
    - RDF (maybe, depending on other groups?)
    - Ticket system with system status
    - Benchmarking could be interesting

