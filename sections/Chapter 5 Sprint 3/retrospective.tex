\section{Retrospective}

\subsection{Reflection on the Sprint}
All features planned during the sprint planning phase were implemented by the end of the sprint, but we did not document the code as well as we had initially planned. 
This clashes with our Definition Of Done, which means we did not fully reach the goal of this sprint. We have attached our Definition of Done in section \ref{sec:definitionOfDone}.

We planned out the implementation to a certain degree, but used a non-optimal format for the tasks, which led to confusion down the line. Namely, we should have described what the task encapsulated more than giving it a simple name like 'POST'. 

We also did not use user-stories, which added to the confusion, as they would have otherwise provided a certain context for the tasks.

The remaining tasks were lackluster and-or nonexistent, which led to trouble doing structured writing, as we had to figure out what should be written every time we wanted to write.
    
To some extent, this was caused by conflicting information given by various sources. We also received a lot of 'urgent', but not important' tasks from other groups, which we had to react to immediately.
    
Due to the above, there were times when individual members did not know exactly what we had finished nor what other members were working on. Tasks were added almost daily, continuously changing the scope of the sprint.
As such, we did not hit our original point-estimate---it became completely irrelevant due to the rapid changes in scope.

\subsection{What We Learned}\label{Whatwelearnedsprint3}
We learned that we had to be more vigilant about narrowing the scope of our tasks. 
This was especially true when working on the APIs where we initially thought that we should develop full CRUD capabilities, but it turned out that only some of the capabilities were necessary. 
From this, we gathered that we had to increase communication efforts with the other groups, as to better understand their needs and wants. 

We should not change response object structure after it has been deployed---we could solve this by implementing classes that represent the response objects instead of sending domain models. 

Intergroup communication was a big part of the sprint, as we had to work with other groups to get the right information. 
Working together while testing gave great insight into the needs of the other groups, and we were able to implement the desired features much quicker.

As the semester courses taught us more about agile methods, we found that we did a few things in a non-conventional way. Planning poker was a great way to get a sense of the scope of the sprint, but we should have used a more structured way to do it. 
Instead of the way we played in during sprint planning, we will make the following optimizations:
Each participant has a set of cards with numbers on. These numbers represent the relative sizing of a task, which is why it is common to use the Fibonacci sequence, where the numbers are well spread out.
Each round begins with the Product Owner reading a user story, which is then discussed. Then, each participant selects a card with their estimate of the task. Once every participant has selected a card, the cards are turned over. If everyone is within two cards of each other, the average of the numbers on the cards is taken as the user story estimate. If that is not the case, the outliers will discuss, and a new round commences.
This is done until there are no more user stories left\cite{sutherlandScrumArtDoing2014}.


We also learned that doing velocity analysis is a great way to know if team productivity is actually improving, which should be a result of us implementing optimizations and removing impediments.
Tasks should also not be marked as complete before they have been shipped to production, as per Scrum convention---a task is not done before it is delivered\cite{sutherlandScrumArtDoing2014}.

\subsection{How We Can Improve in the Next Sprint}

\subsubsection{Start doing}

\begin{itemize}
    \item Create user stories based on input from stakeholders.
    \item Specification of the data in the database---this should be given to the other groups so fewer issues regarding CRUD operations occur.
    \item Update the Kanban board frequently, so group members always are updated
    \item Change planning poker strategy:
    \subitem If everyone is within two cards of each other, just average the sum of your cards.
    \subitem If people are more than three cards apart, the high and low cards talk about why they think what they do.
    \subitem Don't estimate in hours, estimate in 'work' (relative size).
    \item Velocity analysis.
    \item Component diagram versioning---we want to show the progress!
    \item Tasks are not done before they are released. We will make a lane on our Kanban board as a landing for them while they wait for deployment.
    \item Sprints changed to two-week duration. Four weeks is too long.
\end{itemize}

Having concluded the sprint, we will proceed to the next sprint. As mentioned earlier, we will start by documenting what we did not document during this sprint. 
Other tasks will be defined during the sprint planning process, which will be discussed in the next chapter.