\subsection{Continuous Integration and Delivery}\label{CI/CD}
In our work, we have used both Continuous Integration (CI) and Continuous Delivery (CD).
For CI, we have used GitHub with GitHub Actions. We have created a GitHub Actions workflow that
runs the tests in our project; and another one that tests the code quality. These checks run every time commits are made towards the main branch. This ensures that we keep that branch production-ready
at all times.

For CD, we have implemented an automatic deployment system that deploys the code to the production server.
We start the workflow by tagging the commit with the version number. Once we push the tag, a GitHub Actions
workflow is triggered that deploys the code to the production server.
To facilitate the deployment, we containerized our software. The Action workflow starts by building the
Docker image, which is pushed to the GitHub Package Repository. On the production server, we have a Docker 
container watching for new releases. When a new release is pushed, the container pulls the new image and
deploys it.
