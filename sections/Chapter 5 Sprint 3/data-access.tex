\subsection{Data access technologies}
In this section we will give an overview of the data access technologies that we had in consideration for the project.


\textbf{Dapper}
Dapper is an open source Micro ORM or Micro object relational mapper. This means that dapper is a mapping framework that helps mapping the native out queries to domain classes. Dapper is a light weight framework made for developers that prefers to use stored procedures or native query language over using a large scale ORM tool. 

Dapper it self is a NuGet library that when added to projects extendeds the IdbConnection inferface. This extension add helpers for executing a command with no return results, executing a command multiple times, executing a query and mapping it to a strongly typed list and mapping it to a list of dynamic objects. \cite{Dapper_Git}

Dappers simple approach to ORM's means that a large part of the stadard features have been dropped. This means that Dappers focuses on being a lightweight and efficient framework that should cover most of the users needs over being a fullfledge ORM.\cite{Dapper_Git} 

This means that Dapper as a framework covers our usecases with the caveat that additional time is going to be needed in order to fullfil the sprints mvp. 

\textbf{Entity Framework Core}
The second technology that we considered was Entity Framework Core (EF Core), which is a data access technology developed by Microsoft for the .NET platform. EF Core can serve as an object-relational mapper (https://en.wikipedia.org/wiki/Object\%E2\%80\%93relational_mapping) which lets developers abstract away the complications of converting data to .NET objects, in addition to eliminating the majority of the data-access code that would typically have to be written. (https://docs.microsoft.com/en-us/ef/core/).
EF Core works by using models to create the data access, where a model represents the entities in a database and a context object representing a session with the given database. Queries are then able to be performed through the context object. 
This means that EF Core serves as a layer between 

% What are the alternatives?
% - ADO was an option. Why/Why not?
% - Dapper was an option. Why/why not?
% - EF core was an option. Why/Why not?
% Why EF core worked for us
% - General overview of what EF core is

