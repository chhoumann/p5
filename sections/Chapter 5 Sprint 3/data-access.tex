\subsection{Data access technologies}
In this section we will give an overview of the data access technologies that we had in consideration for the project.


\textbf{Dapper}
\textbf{Entity Framework Core}
The second technology that we considered was Entity Framework Core (EF Core), which is a data access technology developed by Microsoft for the .NET platform. EF Core can serve as an object-relational mapper (https://en.wikipedia.org/wiki/Object\%E2\%80\%93relational_mapping) which lets developers abstract away the complications of converting data to .NET objects, in addition to eliminating the majority of the data-access code that would typically have to be written. (https://docs.microsoft.com/en-us/ef/core/).
EF Core works by using models to create the data access, where a model represents the entities in a database and a context object representing a session with the given database. Queries are then able to be performed through the context object. 
This means that EF Core serves as a layer between 

% What are the alternatives?
% - ADO was an option. Why/Why not?
% - Dapper was an option. Why/why not?
% - EF core was an option. Why/Why not?
% Why EF core worked for us
% - General overview of what EF core is

