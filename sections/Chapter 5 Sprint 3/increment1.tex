\section{Increment 1.}
This section will cover the progress made in the first increment. 

% --TEORI--

% Docker
    % hvad kan det
    % hvorfor bruger vi det
\subsection{Docker}
Docker is an open source containerization platform that allows for a simplification of deliveries in distributed applications through the usage of containers. 
Before giving a more detailed description of Docker, an overview of what a container is and how it is used is needed. 


Containers were originally developed to solve the issue where a program may work on one system but encounter problems when moved to a different one. 
Using containers allows developers to package their code together with its required dependencies. 
Moving around a prepackaged application with all its dependencies ensures that the software is going to run the same regardless of the infrastructure in place. 


Here a container image is a lightweight version encapsulating everything needed to run that application. These images are then turned into containers at runtime. 
Mowing around a prepackaged application with all its dependencies insures that the software is going to run the same regardless of the infrastructure in place. 
This is made possibly by the built process isolation and virtualization capabilities in the Linux kernel. 
These capabilities allows for multiple application components to share the ressources of single host operating system, 
in much the same way a hypervisors allows multiple virtual Machines to share the same hardware resources of a single computer. 
\todo{Cite docker og IBM}


The result being that containers allows for the same functionality and benefits as VM’s gives us the following advantages:

\begin{itemize}
    \item Light weight 
    \item Resource efficiency
    \item Improved developer productivity
\end{itemize}

Docker itself it then used to enhance these native linux features allowing us to easier move the containers between environments and automation of container creation.

