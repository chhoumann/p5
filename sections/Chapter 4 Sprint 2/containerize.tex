\section{Containerizing}\label{sec:Containerizing}
Once the data access technology was decided on, the next step for us was to consider whether to put the WordCount database inside a Docker container, as we had originally decided to do with Virtuoso. A benefit of using Docker is that we could easily set up a local development database environment that had the required dependencies on all our computers. Using Docker also meant that we would be able to set up proper encapsulation due to the way access can be managed. This was one of the key requirements that we had, as we wanted to streamline how the \knox{} project performed operations on the database by setting up an API. 
Similarly, we wanted to containerize the aforementioned API as well. That way, we could bridge the API and database containers through a network such that the only access to the database would be through said API.
The added benefit of setting up the database and the API in Docker was that the previously mentioned structure could easily be set up on the university servers once we wanted to go into production.
Unfortunately, due to the lack of time remaining in this sprint, we had to push the development of this into the third sprint.