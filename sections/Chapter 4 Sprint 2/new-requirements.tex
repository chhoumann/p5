\section*{New requirements(Working title)}

During the sprint it was realized that the inital requirements from the \knox{} project had been too vague in their definition. This lead to a lack of a shared understanding between the groups, the product owner and the supervisors, as to what an operational search engine entailed. For our group the initial understanding of the requirement was that the existing API's were sufficient and therefore our attention turned to understanding Virtuoso despite our intention to rewrite the original code to C#. 
As the sprint progressed the requirements for the \knox{} project changed and became significantly more specific and as a result the requirements for the database layer changed. It became clear that virtuoso was not a requirement at this stage, but extending the services for the wordcount database was. As a result, we decided that this was the moment to rewrite the original API's from Java to C#.
The first step in this process was to decide on which data access technologies to use.


% po siger de skal kunne hente data og indsætte data
% vejleder 

% timeline 
% vejleder siger kig på virtuoso
%   vi går ud fra at search engine virker
% PO siger search engine skal virke
%   man kunne søge og hente i DB
%   vi antager det virker for vores lag
% Krav var uspecifikke
% nyt møde med skarpere krav
%   vi skifter fokus til wordcount 
%   vi går i gang med api'er