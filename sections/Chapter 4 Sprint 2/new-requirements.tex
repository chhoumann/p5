\section{New Requirements} \label{ssec:newRequirements}
During the sprint, it was realized that the initial requirements from the \knox{} project had been too vague in their definition. 
This led to a lack of a shared understanding between the groups, the product owner and the supervisors about what an operational search engine entailed.  
We believed that we could prioritize the RDF database over the search engine, as the existing endpoints were largely satisfactory for the purposes of the project. This belief came from discussions with other groups after our sprint planning session. The application layer told us that they were able to fetch test data from the WordCount database. The knowledge layer told us that they seemingly were able to insert data as well.
However, test data and 'seemingly' was not good enough for the \knox{} PO.
As the sprint progressed, the requirements for the \knox{} project changed and became significantly more specific and, as a result, the requirements for the database layer changed.
It became clear that Virtuoso was no longer a requirement at this stage, but extending the services for the WordCount database was.
As a result, we decided to redirect our attention to our initial goal, which was to rewrite the original codebase from Java to C\# and add the required extra features.

In order to successfully transition the codebase into C\#, an impediment was to first decide on which type of data access we wanted to use.
