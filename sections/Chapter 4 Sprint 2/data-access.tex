\section{Object Relational Mapping}\label{ORM}
In order to improve on the way data access was done in the original code base, we decided to consider Object Relational Mapping (ORM).
An ORM is often used in an application to connect to and manipulate data in a relational database\cite{ORM}. 

Additionally, the decision to use an ORM was based on the need for fast implementation and ease of use due to the timeline constraints. This constraint was partly imposed by the \knox{} project due to a change of direction from the updated goal specification and partly due to our desire to rewrite the codebase in C\#. 

\subsubsection{Dapper}
There are a variety of different ORMs that work within C\#, but we were mostly familiar with two technologies. The first of these being Dapper.
 
Dapper is an open source Micro ORM. It is a mapping framework that helps map the result from a native query to a class with the same attributes as the tables. Dapper is a lightweight framework made for developers that emphasizes using stored procedures or native query language over using a large scale ORM tool. 


Dapper, when added to projects, extends the \texttt{IDbConnection} interface. This extension adds helper methods. 
One of the helpers is the \texttt{Query} method. This method will take an SQL query as a parameter and return a list. The returned list can either be strongly or dynamically typed.
Another helper that is added to \texttt{IDbConnection} is the \texttt{Execute} method. This method takes one or more SQL commands as a parameter and has no return type. When called, the commands passed to the method will be executed. The \texttt{Execute} method can also be used to execute commands multiple times\cite{Dapper_Git}.

Dapper's simple approach to ORMs means that a large part of the standard features have been dropped. This means that Dapper focuses on being a lightweight and efficient framework that should cover most of the user's needs over being a full-fledged ORM\cite{Dapper_Git}.

This means that Dapper as a framework covers our use cases, with the caveat that additional time is going to be needed in order to fulfill the sprint MVP. 

\subsubsection{Entity Framework Core}
The second technology that we considered was Entity Framework Core (EF Core), which is a data access technology developed by Microsoft for the .NET platform. EF Core can serve as an object-relational mapper\cite{Object_relational_mapping} which lets developers abstract away the complications of converting data to .NET objects in addition to eliminating the majority of the data-access code that would typically have to be written.

EF Core uses models to create the data access. A model is composed of entity classes and a context object representing a database session.
Queries are then able to be performed through the context object. 
The use case for EF Core is that it is a fast way of implementing data access without the need for as much manual implementation as is the case with Dapper. 
The drawback of EF Core is that the learning curve is steep, and one must be at least an intermediate level user of the framework to avoid common security and performance pitfalls\cite{EFCore}.

Due to previous experience with the framework among group members, we were able to quickly implement EF Core and fulfill the use cases that we had.
With this in mind, we concluded that we were better able to accommodate the tight deadline using EF Core to fulfill the required features for the \knox{} project.