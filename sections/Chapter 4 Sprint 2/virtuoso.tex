\section{Virtuoso}
In this sprint, we had planned to research and implement Virtuoso as per our supervisors recommendation.
The research and implementation did get cut a bit short because the priorities of the \knox{} project changed on request of the PO.

During our research, we found that Virtuoso is a high-performance object-relational SQL database, which is highly effective at storing and querying RDF data. It supports SPARQL queries embedded in SQL queries. SPARQL is an acronym for SPARQL Protocol and RDF Query Language, which is an RDF query language.


Virtuoso supports access to the database through Open DataBase Connectivity (ODBC).
ADO.NET is an older framework for C\# which can give database access to databases that are ODBC compliant.


After the initial research, we attempted to set up a Virtuoso server on a Virtual machine. This was a partial success, we created a database and were able to access it through the GUI in a browser, but we were not able to make HTTP requests to the server.

The next step would be to set up Virtuoso and an API for it, in a development environment using Docker.