\section{Retrospective}

\subsection{Reflection on the sprint}
This sprint was quite a learning curve for our group, as the structure and communication between the \knox{} groups was disorganized. Especially during the initial phases of the sprint. 
Our understanding of the requirements from the \knox{} project turned out to be different from what may have initially been intended, but we only learned this fact much later into the sprint. 
This meant that we had to adapt quickly to the situation and change our goals to match the new requirements. 
This adaptation also meant that we decided upon planning to update the inherited code base from the previous semester sooner than we had originally intended. 
It also meant that what we had originally researched became redundant for the time being and our intentions with Docker also changed to include the WordCount database as well as an API end-point for this, rather than the Virtuoso Database. 
The conclusion of the sprint ended up being that we had to push the implementation of these until the third sprint.

\subsection{What we learned}
We learned that we had to spend more time making sure that the tasks and requirements were fully understood before proceeding. 
A shortcoming from our perspective was that we did not inquire enough about the specifics of the goal that had been set for us initially and as a result, much of our work was based on an assumption. 
What we did right, was our usage of agile development strategies and methodologies once we learned that the goal had been adjusted. 
We were quickly able to reflect on our shortcomings and followed it up by implementing better organization within the group. 
Some of these implementations included appointing a product owner within our group to make tasks, appointing a scrum master to handle the daily planning and ensure that the scrum methods were followed. 
Making these changes meant that we had a more streamlined process with the overall planning and execution of tasks, and our communication internally improved as a result.

\subsection{How we can improve in the next sprint}
\subsubsection{Start doing}
\begin{itemize}
    \item Spend more time understanding the exact tasks for the given sprint.
    \item If you implement something new, a presentation should be made for the entire group, to get everyone up to speed.
    \item Keep better track of the different boards (Notion and ClickUp).
\end{itemize}
\subsubsection{Continue doing}
\begin{itemize}
    \item Agile methods.
\end{itemize}

Having concluded the sprint, we will proceed to the next sprint.