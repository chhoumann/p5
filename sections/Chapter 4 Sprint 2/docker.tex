\section{Docker}
To set up our development environment for an API and Virtuoso, we decided to use Docker, which is an open source containerization platform.

Docker allows for a simplification of deliveries in distributed applications through the usage of containers\cite{Container_Docker}. 
Before giving a more detailed description of Docker, an overview of what a container is and how it is used is needed. 


Containers were originally developed to solve the issue where a program may work on one system but encounter problems when moved to a different one. 
Using containers allows developers to package their code together with its required dependencies. 
Moving around a prepackaged application with all its dependencies ensures that the software is going to run the same regardless of the infrastructure in place\cite{Container_Docker}.


A container image is a lightweight, standalone, executable package encapsulating everything needed to run an application. 
These images are then turned into containers at runtime.
This is made possible by the build process isolation and virtualization capabilities in the Linux kernel. 
These capabilities allow for multiple application components to share the resources of a single host operating system in much the same way a hypervisor allows multiple virtual machines to share the same hardware resources of a single computer\cite{Container_Docker}.


Using Docker rather than a virtual machine provides the following advantages:

\begin{itemize}
    \item Lighter weight
    \item Resource efficiency
    \item Improved developer productivity
\end{itemize}

Docker itself is then used to enhance these native Linux features, allowing us to automate container creation and easily move the containers between environments\cite{Docker_IBM}.