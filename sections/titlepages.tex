\pdfbookmark[0]{English title page}{label:titlepage_en}
\aautitlepage{%
  \englishprojectinfo{
    Knox %title
  }{%
    Scientific Theme %theme
  }{%
    Fall Semester 2021 %project period
  }{%
    cs-21-sw-5-19 % project group
  }{%
    %list of group members
    Christian Bager Bach Houmann\\
    Daniel Overvad Nykjær\\
    Ivik Lau Dalgas Hostrup\\
    Marco Klaustrup Justesen\\
    Patrick Frostholm Østergaard\\
    Rasmus Høyer Hansen
  }{%
    %list of supervisors
    Christian Aebeloe
  }{%
    1 % number of printed copies
  }{%
    \today % date of completion
  }%
}{%department and address
  \textbf{Computer Science}\\
  Aalborg University\\
  \href{http://www.aau.dk}{http://www.aau.dk}
}{% the abstract
Knox (Knowledge Engineering Toolbox) is a continuous pipeline project involving several project groups working in a scrum-inspired environment.
This report documents the work of the third layer of the pipeline, responsible for database and server management.
For the first half of the project, the groups collaborated on developing a search engine.
Here, we designed and implemented database structures in the relational database management system PostgreSQL.
Using HTTP, the other layers can then access the API endpoints for the search engine database.
For the second half of the project, the groups began development of features based on knowledge graphs.
Here, we supported the other groups by implementing a data storage for the used RDF data. 
}

% \cleardoublepage
% {\selectlanguage{danish}
% \pdfbookmark[0]{Danish title page}{label:titlepage_da}
% \aautitlepage{%
%   \danishprojectinfo{
%     Rapportens titel %title
%   }{%
%     Semestertema %theme
%   }{%
%     Efterårssemestret 2010 %project period
%   }{%
%     XXX % project group
%   }{%
%     %list of group members
%     Forfatter 1\\ 
%     Forfatter 2\\
%     Forfatter 3
%   }{%
%     %list of supervisors
%     Vejleder 1\\
%     Vejleder 2
%   }{%
%     1 % number of printed copies
%   }{%
%     \today % date of completion
%   }%
% }{%department and address
%   \textbf{Elektronik og IT}\\
%   Aalborg Universitet\\
%   \href{http://www.aau.dk}{http://www.aau.dk}
% }{% the abstract
%   Her er resuméet
% }}
% 