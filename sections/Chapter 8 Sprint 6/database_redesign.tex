\section{Redesigning the WordCount Database}
In section \ref{InitialDesign}, we established that the database design would lead to duplicate or redundant data in the WordCount database.
Namely, the terms occurring in the Term entity set (see section \ref{newdatabaseER}) will not be unique. 
Thus, the database can contain duplicate values.
 
To address this issue, we created relations using the relational model (see section \ref{relational_databases}) and ensured they are well-designed.
After ensuring that a normal form is reached, we created the relations using SQL as a DDL.
We also needed to adjust the data access models (see section \ref{models}) to ensure access using EF core.
When implementing the new database structures, we needed to ensure that the pipeline is operational and has minimum downtime. 
Therefore, we had to ensure we could always roll back changes to the code and database structures.
As described in section \ref{CI/CD}, we can easily deploy versions of our code.
By reverting the changes to the old implementation of the models, we were able to ensure a quick response in case the new implementation failed.
We needed to ensure that this was also the case for the database structures.
This was done by taking a dump (backup) of the database, and verifying that no database relations are deleted during the data migration.

\subsection{Redesigning the Database Model}\label{databaseModelRedesignNF}
We will now create relations describing the new models for the database.
Other groups of the \knox{} pipeline have requested that entities representing articles have a unique Id field, regardless of what is considered good database design.
Similarly, they want a field denoting the total words appearing in the article.

We will start by creating an ER-model of the domain, and then describe the entity and relationship sets using the relational model, using the method described in section \ref{relational_databases}.

\begin{figure}[H]
    \centering
    \includegraphics[scale=0.35]{Images/new ER.drawio.png}
    \caption{ER diagram depicting the new model of the database. Notice that the relationship set \textit{Publishes} and entity set \textit{publisher} has not changed.}
    \label{fig:newdatabaseRedesignER}
\end{figure}

Equation \ref{eq:newDatabaseRelationalModel} shows relations representing the new database structures from figure \ref{fig:newdatabaseRedesignER}.
Notice that the existing $publisher$ relation in the database has exactly the same attributes as the new one. 

\begin{equation}\label{eq:newDatabaseRelationalModel}
    \begin{split}
        article(\underline{article\_id: \mathbb{Z^+}}, title:text,\\ total\_words:\mathbb{Z^+}, publisher\_name \rightarrow publisher, file\_path:text), \\
        word(\underline{text:text}),\\
        publisher(\underline{publisher\_name:text}),\\
        occursIn(\underline{text \rightarrow word}, \underline{article\_id \rightarrow article}, count:\mathbb{Z^+})\\
    \end{split}
\end{equation}


We have to ensure that the WordRatio view, described in section \ref{WordRatioCrud} and appendix \ref{Appendix_WordRatioOld}, still contains the same information as before.
Since the view is an abstraction over a \texttt{SELECT} query, we will describe it using relational algebra.
The query of the view can be seen in equation \ref{create_new_view_rel_alg}.
The resulting relation structure is seen in equation \ref{create_new_view_rel_alg_structure}.


\begin{equation}\label{create_new_view_rel_alg}
    \begin{split}
        \sigma_{article\_id, text, count,title, file\_path, total\_words, publisher\_name, percent} \\
        ((article \Join_{article.article\_id = occursIn.article\_id} occursIn) \\
        \Join_{article.publisher\_name = publisher.publisher\_name} publisher\\
        \Join \gamma_{article, occursIn;percent(count, total\_words)})
    \end{split}
\end{equation}

\begin{equation}\label{create_new_view_rel_alg_structure}
    \begin{split}
        wordRatio(article\_id:\mathbb{Z^+}, text:text, count:\mathbb{Z^+},title:text, file\_path:text,\\ total\_words:\mathbb{Z^+}, publisher\_name:text, percent:\mathbb{R^+})
    \end{split}
\end{equation}

\subsubsection*{Ensuring a normal form}
Having established the structure of the new database schemas, we can now examine whether or not they live up to TNF or BCNF (see section \ref{relational_databases}).
This is done by examining the functional dependencies of the relations seen in equation \ref{eq:newDatabaseRelationalModel}.

\begin{equation}\label{eq:newDatabaseRelationalModel}
    \begin{split}
        article(\underline{article\_id: \mathbb{Z}^+} title:text, totalwords:\mathbb{Z}^+, publisher\_name \rightarrow publisher), \\
        word(\underline{text:text}), \\
        publisher(\underline{publisher\_name:text}), \\
        occursIn(\underline{text \rightarrow word}, \underline{article\_id \rightarrow article}, count:\mathbb{Z}^+) \\
    \end{split}
\end{equation} 
The \textit{occursIn} relation is in BCNF. 
This is verified when examining the functional dependency of its composite primary key:
\begin{equation*}
 \{text,article\_id\}\rightarrow \{count\}   
\end{equation*}

A unique article cannot identify the text and number of times the text occur in the article.
Similarly the number of times the text occurs in the article and the article\_id cannot identify the text of the word.

The \textit{article} relation is in BCNF form. 
This is seen when considering the functional dependency of its primary key:
\begin{equation*}
    \{article\_id\} \rightarrow \{title, publisher\_name,totalwords\}
\end{equation*}

One could consider using only title as key. 
However, some article titles could occur multiple times; this is realized when considering the domain area of the \knox{} search engine
\footnote{For instance the title "The first snow of the season" could be a occurring every year for the same publisher.}.

The \textit{word} and the \textit{publisher} relation is in BCNF trivially:
\begin{equation*}
    \{ text\} \rightarrow \{text\} \textit{\ and\ } \{ publisher\_name \} \rightarrow \{ publisher\_name\}
\end{equation*}


\subsubsection*{Implementing the Models}\label{implementing_new_WC_DB_models}
After having established the quality of the design of the database structures, we then implemented them.
When creating the new tables, we ensured that no loss of data occurred by not dropping the old tables and instead renamed them. 
The tables could then be dropped later after we completed the implementation of the new design.
The renaming of the tables can be seen in code snippet \ref{lst:RenamingOldModelsSQL}.

\begin{lstlisting}[
    label=lst:RenamingOldModelsSQL,
    language=SQL,
    caption=SQL script renaming the existing database structures.,
    showspaces=false,
    numbers=left,
    numberstyle=\footnotesize,
    commentstyle=\color{gray},
    escapechar=|
 ]
 ALTER TABLE "Article"
    RENAME TO "Article_Redundant";
ALTER TABLE "Term"
    RENAME TO "Term_Redundant";
\end{lstlisting}

When implementing the new tables, we based them on the newly renamed tables from the old database design.
By doing this, we ensured that all data from the old tables were correctly transferred to the new ones, since each command is a transaction (see section \ref{relational_databases}).
After all data had been transferred to a new table, constraints, foreign- and primary keys were added to the tables. 
This can be seen in code snippet \ref{lst:CreatingTheNewDBModelsSQL}.
We have ensured that the changes are reversible by creating a script dropping the new tables, and restoring the structure of the old tables.
The script is found in Appendix \ref{SQLBackupScript}.

\begin{lstlisting}[
    label=lst:CreatingTheNewDBModelsSQL,
    language=SQL,
    caption=SQL script creating the tables described in \ref{databaseModelRedesignNF}. Foreign- and primary key constraints are also defined.,
    showspaces=false,
    numbers=left,
    numberstyle=\footnotesize,
    commentstyle=\color{gray},
    escapechar=|
 ]
CREATE TABLE "Article" AS
SELECT * FROM "Article_Redundant";
ALTER TABLE "Article"
    ADD CONSTRAINT "Id_PRMK" PRIMARY KEY ("Id"),
    ADD FOREIGN KEY ("PublisherName") 
        REFERENCES "Publisher"("PublisherName");

CREATE TABLE "Word" AS
SELECT DISTINCT t."Word" AS "Text" FROM "Term_Redundant" t;
ALTER TABLE "Word" ADD CONSTRAINT "Literal_PRMK" PRIMARY KEY ("Text");

CREATE TABLE "OccursIn" AS
    (
        SELECT "art"."Id" as "ArticleId", "t"."Count",  t."Word"
        FROM "Term_Redundant" t JOIN wordcount.public."Article_Redundant" 
            art ON t."ArticleId" = art."Id"
    );
ALTER TABLE "OccursIn"
    ADD FOREIGN KEY ("ArticleId") REFERENCES "Article"("Id"),
    ADD FOREIGN KEY ("Word") REFERENCES "Word"("Text"),
    ADD PRIMARY KEY ("ArticleId", "Word");
\end{lstlisting}

After having implemented the new database structures, we redefined the WordRatio view.
Notice that the view is an abstraction over a \texttt{SELECT} statement, thus it is replaceable without loss of data.


\begin{lstlisting}[
    label=lst:DefiningNewView,
    language=SQL,
    caption=SQL script defining a new WordRatio view.,
    showspaces=false,
    numbers=left,
    numberstyle=\footnotesize,
    commentstyle=\color{gray},
    escapechar=|
 ]
CREATE OR REPLACE view "WordRatio"("ArticleId", "Word", "Count", "Title", "FilePath", "TotalWords", "PublisherName", "Percent") as
SELECT "OccursIn"."ArticleId",
       "OccursIn"."Word",
       "OccursIn"."Count",
       "Article"."Title",
       "Article"."FilePath",
       "Article"."TotalWords",
       "Article"."PublisherName",
       round("OccursIn"."Count"::numeric 
            / "Article"."TotalWords"::numeric * 100::numeric, 2) 
            AS "Percent"
FROM "Article"
         JOIN "OccursIn" ON "OccursIn"."ArticleId" = "Article"."Id"
         JOIN "Publisher" ON "Article"."PublisherName" 
            = "Publisher"."PublisherName";
\end{lstlisting}

\subsection{Result}
After having implemented the new database structure, we assessed the changes to the database. 
The number of words in the newly implemented \texttt{Word} relation is 260,000.
The number of unique tuples in the newly implemented \texttt{WordOccurance} table is 6,900,000, the same as the number of tuples representing words in the old \texttt{Term} relation.
Thus, the number of tuples in the new relation is the same, but the number of duplicate words have been reduced greatly.
