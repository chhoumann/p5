\section{Sprint Planning}
During this sprint, we will be focusing on creating a functioning UI component, redesigning the database schemas to minimize the amount of stored data as well as creating a video describing the flow of our endpoints. Furthermore, we would like to deliver the knowledge graph components from last sprint. 

\subsection*{User stories}
This sprint will focus on two user stories provided to us by the \knox{} product owner as well as one arranged by our own product owner. 
Contrary to the last sprint, these user stories are not seen as epics and are therefore added directly to the sprint backlog. 

\begin{enumerate}
    \item As a future \knox{} member, I want a video describing the flow of the WordCount database endpoint. 
    \item As the database layer, I want to redesign the database schemas to minimize the size of the data stored. 
    \item As the database layer, I want to develop a simple UI component denoting the database status such that I can fulfil my \knox{} assignments. 
\end{enumerate}

A game of planning poker was then played in order to provide an estimation of the size of the tasks. 
Using this, we were able to decide which tasks we could bring over from the previous sprint.

\subsection*{Release planning}
As we do not control the live branch for the deployed \knox{} website, the acceptance criteria for the UI component is creating a pull request with a functioning component and having that request accepted.
This procedure was establish by the UI committee. 
For the WordCount database redesign, as well as the knowledge graph component started last sprint, the standard deployment procedure is expected. 

With the sprint planned, the development phase can now start.
