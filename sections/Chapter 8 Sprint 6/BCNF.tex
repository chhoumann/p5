\subsubsection*{Quality of the New Design}
Upon having established the structure of the new database schemas, we then examined whether they are in TNF or BCNF (see section \ref{relational_databases}).
This was done by examining the functional dependencies of the relations seen in equation \ref{eq:newDatabaseRelationalModel}.

The \textit{occursIn} relation is in TNF. 
This is verified when examining the functional dependency of its composite primary key:
\begin{equation*}
 \{text,article\_id\}\rightarrow \{count\}   
\end{equation*}

A unique article cannot identify the text and number of times the text occurs in the article.
Similarly, the number of times the text occurs in the article and the $article\_id$ cannot identify the text of the word.

The \textit{article} relation is in TNF. 
This is seen when considering the functional dependency of its single, primary key:
\begin{equation*}
    \{article\_id\} \rightarrow \{title, publisher\_name,total\_words\, file\_path\}
\end{equation*}

One could consider using only \textit{title} as key. 
However, some article titles could occur multiple times; this is realized when considering the domain area of the \knox{} search engine.
For instance, the title "The first snow of the season" could occur every year for the same publisher.
Thus, the composite, primary key $\{ publisher,article\_title \}$ would not guaranteee uniqueness either.
One should notice that each of the files in the search engine (represented by the $file\_path$ attribute) can contain multiple articles.
An example of a $file\_path$ could be a path to a newspaper in PDF format, containing many articles about the weather.
Therefore neither $\{file\_path\}$ nor $\{ file\_path, article\_title \}$ guarantee unique primary keys.


The \textit{word} and the \textit{publisher} relations are in BCNF trivially:
\begin{equation*}
    \{ text\} \rightarrow \{text\} \textit{\ and\ } \{ publisher\_name \} \rightarrow \{ publisher\_name\}
\end{equation*}
