\subsubsection*{Quality of the new design}
Having established the structure of the new database schemas, we can now examine whether they live up to TNF or BCNF (see section \ref{relational_databases}).
This is done by examining the functional dependencies of the relations seen in equation \ref{eq:newDatabaseRelationalModel}.

The \textit{occursIn} relation is in BCNF. 
This is verified when examining the functional dependency of its composite primary key:
\begin{equation*}
 \{text,article\_id\}\rightarrow \{count\}   
\end{equation*}

A unique article cannot identify the text and number of times the text occurs in the article.
Similarly, the number of times the text occurs in the article and the article\_id cannot identify the text of the word.

The \textit{article} relation is in BCNF form. 
This is seen when considering the functional dependency of its single, primary key:
\begin{equation*}
    \{article\_id\} \rightarrow \{title, publisher\_name,total\_words\, file\_path\}
\end{equation*}

One could consider using only title as key. 
However, some article titles could occur multiple times; this is realized when considering the domain area of the \knox{} search engine
\footnote{For instance, the title "The first snow of the season" could be a occurring every year for the same publisher.}.
One should notice that each of the files in the search engine (represented by the $file\_path$ attribute) can contain multiple articles\footnote{An example of a $file\_path$ could be a path to a newspaper in PDF format, containing many articles about the weather.}.
Thus, the $file\_path$ attribute cannot uniquely identify articles.
Similarly, a PDF file might have chapters (articles) with the same name. For instance, "Review" or "Retrospective".
Thus, using a composite key of \textit{article\_id, file\_path} is not viable. 

The \textit{word} and the \textit{publisher} relation is in BCNF trivially:
\begin{equation*}
    \{ text\} \rightarrow \{text\} \textit{\ and\ } \{ publisher\_name \} \rightarrow \{ publisher\_name\}
\end{equation*}

