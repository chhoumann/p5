\subsubsection*{Ensuring a normal form}
Having established the structure of the new database schemas, we can now examine whether or not they live up to TNF or BCNF (see section \ref{relational_databases}).
This is done by examining the functional dependencies of the relations seen in equation \ref{eq:newDatabaseRelationalModel}.

\begin{equation}\label{eq:newDatabaseRelationalModel}
    \begin{split}
        article(\underline{article\_id: \mathbb{Z}^+} title:text, totalwords:\mathbb{Z}^+, publisher\_name \rightarrow publisher), \\
        word(\underline{text:text}), \\
        publisher(\underline{publisher\_name:text}), \\
        occursIn(\underline{text \rightarrow word}, \underline{article\_id \rightarrow article}, count:\mathbb{Z}^+) \\
    \end{split}
\end{equation} 
The \textit{occursIn} relation is in BCNF. 
This is verified when examining the functional dependency of its composite primary key:
\begin{equation*}
 \{text,article\_id\}\rightarrow \{count\}   
\end{equation*}

A unique article cannot identify the text and number of times the text occur in the article.
Similarly the number of times the text occurs in the article and the article\_id cannot identify the text of the word.

The \textit{article} relation is in BCNF form. 
This is seen when considering the functional dependency of its primary key:
\begin{equation*}
    \{article\_id\} \rightarrow \{title, publisher\_name,totalwords\}
\end{equation*}

One could consider using only title as key. 
However, some article titles could occur multiple times; this is realized when considering the domain area of the \knox{} search engine
\footnote{For instance the title "The first snow of the season" could be a occurring every year for the same publisher.}.

The \textit{word} and the \textit{publisher} relation is in BCNF trivially:
\begin{equation*}
    \{ text\} \rightarrow \{text\} \textit{\ and\ } \{ publisher\_name \} \rightarrow \{ publisher\_name\}
\end{equation*}

