\section{Conclusion}\label{scrumofscrumsConclusion}
\textit{The following sections have been written in collaboration with other Knox project groups.}
\textcolor{blue}{\textit{This section has been made in collaboration between all groups.}} \\

\noindent This sprint concluded the work with regard to getting the search engine up and running in accordance with the MVP. To see if the MVP was met, an end-to-end test of the entire system was planned for October 27th, where each group would send a representative. Each representative had the responsibility of testing their part of the pipeline with real data. 

\subsection{Scrum-of-Scrums Retrospective}\label{scrumofscrumRetrospective}
The retrospective is used for evaluating sprints. The questions are from the book "Scrum: the art of doing twice the work in half the time" \cite[p. 238]{sutherland2014scrum}.\\
\noindent During the intermission, a retrospective meeting was held to evaluate what went well and what did not go well during part one, as well as what can be improved for future sprints. 

\subsubsection{What went well}
The collaboration between Scrum teams improved as time went by. The fact that the groups were physically located close to each other made the communication between groups easy. People also appreciated the planned end-to-end test to identify flaws in the current pipeline and fix them as they were discovered.

\subsubsection{What did not go well}
The first issue during this part was that the presentation of the whole Knox project was not well-defined. This resulted in misunderstandings regarding the goals of the entire Knox project. This meant that the Scrum teams had a hard time, in the beginning, defining \textit{sprint backlog} tasks. From a student's perspective, it also seemed like the supervisors were unclear about the scope of the project. This uncertainty, lack of \textit{backlog} tasks, and misunderstandings led to a summit, which resulted in a new goal defined by the minimum viable product (MVP).\\

\noindent The \textit{product owner} (PO) attended the Scrum-of-Scrums meetings to help clarify confusions. In hindsight, we did not use the PO enough and at times had a hard time prioritizing our tasks. A reason for this could be that groups were so dependent on each other and new tasks kept occurring. \\

\noindent Lastly, even though communication was one of the things that went well for the sprints, it still has to be improved. Miscommunication kept happening in regard to defining tasks and deciding who was responsible for the tasks between the layers, as well as the committees lacking purpose and feeling like they wasted time on their meetings.

\subsubsection{Actions for future sprints}
Goals for the next sprints were better sprint planning and making sure that everybody has a clear understanding of the product goal. 
To help unravel confusions regarding tasks we have to be better at making contact with the PO especially if new tasks arise during a sprint. 
To facilitate communication between the Scrum teams, it was decided that the groups should try to use user stories between the groups. Meaning that, when new tasks are given to a Scrum team, a story on how it should work should be included.
Lastly, the Scrum teams have to be better at communicating whether or not they are on time to reach the sprint goal. If they are not, they should be better at asking for help from other groups. 