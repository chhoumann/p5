
Having established how to convert an ER diagram into an equivalent relational model, we can now describe the operations for the relational model.

\subsubsection{Relational Algebra}\label{sec:relationalAlgebra}
Relational algebra describes a set of unary and binary operations on relations.
The operations are used to define new relations --- often the tuples within the set will satisfy defined predicates.
The relational operators form the foundations of data manipulation languages (see Section \ref{sec:SQL}) which can be used to define database operations \cite{DBSBook}.
A few of these operations can be seen in table \ref{Relational algebra operators}.


\begin{table}[h]
    \centering
    \begin{tabular}{|ll|}
    \hline 
    \multicolumn{1}{|l|}{\textbf{Operator}}          & \multicolumn{1}{l|}{\textbf{Example}}   \\ \hline
    \multicolumn{1}{|l|}{Select}                     & \multicolumn{1}{l|}{$\sigma_{predicate}(R)$}            \\ \hline
    \multicolumn{1}{|l|}{Projection}                 & \multicolumn{1}{l|}{$\pi_{A_1, A2,...,A_n}(R)$}           \\ \hline
    \multicolumn{1}{|l|}{Join}                 & \multicolumn{1}{l|}{$r_1 \Join_\Theta r_2$}             \\ \hline
    \multicolumn{1}{|l|}{Cartesian product}          & \multicolumn{1}{l|}{$r_1\times r_2$}              \\ \hline
    \end{tabular}
    \caption{Table of operators}
    \label{Relational algebra operators}
\end{table}

We will describe some of the operators of relational algebra, and use them to describe implemented queries later.
Some of the common operators of in relational algebra can be seen on table \ref{Relational algebra operators}.
\subsubsection*{Select}
The select operator defines a set, $R_{result}\subseteq R$ where all tuples $t \in R_{result}$ satisfies a given predicate \cite{DBSBook}.
That is, $\forall t \in R_{result} \vDash predicate$.
As an example, we can produce a relation containing all books with the title "Database System Concepts":
$$\sigma_{author\_name = "Database System Concepts"}(books)$$
\subsubsection*{Projection}
The projection operator specifies a relation containing only a subset of the attributes of the operand relation \cite{DBSBook}.
That is, attributes $A_1, ..., A_n$ of relation $R_{result}$ will be a subset or equal to the attributes of the operand.
We can use the projection in combination with the select operator operator, to produce a relation containing only the ISBN and title of books titled "Database System Concepts":
$$\pi_{ISBN,title} (\sigma_{author\_name = "Database\ System\ Concepts"}(books))$$
\subsubsection*{Cartesian product}
The Cartesian product of two relations $R_1$ and $R_2$ produces a set containing concatenated tuples of the two operands.
That is, taking the attributes of the two sets, $A_1,...,A_n \in R_1$ and $A_{n+1},...,A_k \in R_2$ produces a set with attributes $A_1,...,A_k$.
If the attributes of the operands have the same names we distinguish between them  by denoting their original relation: $R_1.AttributeName$, $R_2.AttributeName$. \cite{DBSBook}
For instance, we can concatenate the information in the $book\_owner$ and $owns$  relations.
$book\_owner \times owns = r$ will result a relation with the structure as seen in equation \ref{eq:cart_struct}.
\begin{equation}\label{eq:cart_struct}
    r(owns.owner\_id, owns.ISBN,book\_owner.name,book\_owner.owner\_id, author\_name)
\end{equation}\\    
\textbf{Join}\\
The join operator defines a relation consisting of tuples from the Cartesian product of the operand, containing only tuples satisfying the given predicate.
It is defined as $\sigma_{\Theta} (R_1 \times R_2)$ \cite{DBSBook}.

Thus we can express the relation describing the name of all $book\_owner$ entities, that owns the book with $ISBN$ "987-1-...50-4" as both equation \ref{eq:join_s} and equation \ref{eq:join_j}.
\begin{equation}\label{eq:join_s}
    \pi_{book\_owner.name} (\sigma_{owns.ISBN = 987-1-...50-4}  (book\_owner \times owns))
\end{equation}
\begin{equation}\label{eq:join_j}
    \pi_{book\_owner.name} (book \Join_{owns.ISBN = 987-1-...50-4} owns) 
\end{equation}\\

Other variations of the join operation exists \cite{DBSBook}, however these will not be discussed in this report.
\subsubsection*{Aggregate functions}
In conjunction with the basic relational operators Aggregate functions can be used over a set of values.
Aggregate functions can take a set of values as input and get a single value as output \cite{DBSBook}. Some common aggregate functions are 
\begin{itemize} \label{aggregateFunctions}
    \item \texttt{count} --- returns the number of tuples in the set.
    \item \texttt{sum} --- returns the sum of the values in the set.
    \item \texttt{min} --- returns the smallest value in the set.
    \item \texttt{max} --- returns the largest value in the set.
    \item \texttt{avg} --- returns the average value in the set.
\end{itemize}


As we have now established both how to model database relations as well fundamental operations used to extract data (tuples) from these relations, we can now look at how to implement the relations in a database.

\subsubsection{SQL}\label{sec:SQL}
SQL is one of the most commonly used data definition language (DDL) and data manipulation languages (DML) \cite{DBSBook}.
SQL is a declarative query language, where one executes a query to instruct the database system to perform a set of operations to compute a desired result. 
Collectively, the resulting set of operations is known as a transaction.
In PostgreSQL, all queries, both to define and to manipulate data, are executed as transactions \cite{postgres_transactions}.
As a DDL, SQL provides queries to define and modify relation schemas, as well as provide constraints for the attributes of the schema, such that data in the relation satisfies the constraint.  

\begin{lstlisting}[
    label=lst:CreatingTablesInSQL,
    language=SQL,
    caption=Implementing the relations from in equation \ref{eq:bookOwnerExample} and \ref{eq:relational_schema},
    showspaces=false,
    basicstyle=\ttfamily,
    numbers=left,
    numberstyle=\tiny,
    commentstyle=\color{gray},
    escapechar=|
 ]
    CREATE TABLE book (
        author_name varchar(250) not null,
        title varchar(250) NOT NULL,
        number_of_pages INTEGER NOT NULL CHECK(number_of_pages > 0),
        ISBN char(10) UNIQUE NOT NULL,
        primary key (ISBN)
    );
    CREATE SEQUENCE ownerSequence INCREMENT BY 1 START 1; |\label{line:sequence}|
    CREATE TABLE book_owner (
        name varchar(250),
        owner_id integer NOT NULL DEFAULT nextval('ownerSequence')
    );
    CREATE TABLE owns (
        owner_id INTEGER NOT NULL REFERENCES book_owner(owner_id),
        isbn char(10) NOT null REFERENCES book(isbn),
        PRIMARY KEY (isbn, owner_id)
    );
\end{lstlisting}

Code Snippet \ref{lst:CreatingTablesInSQL} shows a possible definition of the relations from equation \ref{eq:bookOwnerExample} and \ref{eq:relational_schema}.
On line \ref{line:sequence} a sequence describing the \texttt{owner\_id} primary key for the \texttt{owner} relation has been defined.
Whenever a new tuple is inserted into the table, the sequence will generate a unique owner\_id for the data.
As a DML, SQL provides functionality to fetch data stored in the database. 
Many of the operators presented in the previous section is predefined in SQL and describe the exact same operation. 
Code Snippet \ref{lst:join} shows a simple join operation, returning all book titles of books that are owned by the book\_owner with owner\_id 10.

\begin{lstlisting}[
    label=lst:join,
    language=SQL,
    caption=Query returning the title of all books owned by book\_owner with owner\_id = 10,
    showspaces=false,
    basicstyle=\ttfamily,
    numbers=left,
    numberstyle=\tiny,
    commentstyle=\color{gray},
    escapechar=|
 ]
 SELECT title FROM
 owns JOIN book ON owns.isbn = book.isbn WHERE owns.owner_id = 10  
\end{lstlisting}

In relational algebra $\pi_{title}\sigma_{owns.owner\_id = 10}(owns \Join_{book.isbn = book.isbn} book )$.

% Having established how one can define a database schemas using SQL as a DDL, as well as query data from the database using SQL as a DML.