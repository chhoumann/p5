\subsection{Relational Databases}\label{relational_databases}
The group last year chose to implement the WordCount and Fuseki databases using PostgreSQL.
PostgreSQL is a relational database system\cite{knox2020}.

In the coming sections, we will describe how one can model and describe a relational database design, both mathematically and with a more graphic design.
After doing this, we will examine how operations on these models can be described using SQL and relational algebra.

\subsubsection*{Relational Model}
Relational database systems can be mathematically described using relations and sets, mapping a unique key to a tuple of information\cite{DBSBook}.
The values of the tuples contained within the relation can be described by the attributes of the relation and their corresponding domains\cite{KatjaFirstPP}. 
The relations are often described using a \textit{relational schema}, denoting the name and domains of the attributes.


Equation \ref{eq:relational_schema} shows an example of a relation describing books as tuples of three text fields ($ISBN$, $author\_name$, and $title$ ) and a positive integer ($number\_of\_pages$).
The relation also denotes a super key for the relation. A super key is one or more attributes that can uniquely identify a tuple in a given relation.
A super key is called a \textit{candidate key} if removal of a single attribute from the key results in it longer uniquely identifying tuples.
Attributes describing the primary key are underlined.

\begin{equation} \label{eq:relational_schema}
    book(\underline{ISBN: text}, author\_name:text, title: text, number\_of\_pages:\mathbb{Z}^+)
\end{equation}
Super keys can be explained by $t_1 \in r\neq t_2 \in r \implies t_1.K \neq t_2.K$ \cite{DBSBook}, where $t_1, t_2$ are tuples contained within the relation $r$.
That is, no two tuples $t_1, t_2$ from relation $r$ have the same values if they do not have the same superkey $K$. 
If the super key does not contain extraneous attributes, it is said to be \textit{minimal} \cite{DBSBook}.
We will use the term \textit{primary key} to denote a chosen super key of a relation. 
When describing a database, it is often necessary to specify how various data are connected. 
To do this, one can use \textit{foreign keys} to denote that tuples in $r_1$ are related to the tuples in $r_2$.


One could model the relationship between a book owner and a book using the relational schemas seen in equation \ref{eq:relational_schema} and \ref{eq:bookOwnerExample}.
There, primary keys from other relations are used to reference unique tuples. The $owns$ relation describes how relations $book$ and $book\_owner$ are connected. 
\begin{equation}\label{eq:bookOwnerExample}
    \begin{split}
        owns(\underline{owner\_id \rightarrow book\_owner}, \underline{ISBN \rightarrow book}), \\
        book\_owner(\underline{owner\_id:\mathbb{Z}^+}, name:text,)
    \end{split}
\end{equation}

\subsubsection*{Evaluating a Database Design}
When evaluating the design of relational database schemas, we want to avoid redundant data duplication, loss by decomposition and change of dependencies\cite{DBSBook}.
One approach to ensure this is to use normalization theory\cite{DBSBook}. 
This approach examines the \textit{functional dependencies} of the relations and evaluates them based on their \textit{normal form}.
When a relation is in a normal form, it adheres to certain design criteria. 
These criteria are defined using functional dependencies.

A functional dependency describes the relationship between two sets of attributes. 
Functional dependency between $\alpha$ and $\beta$, written as $\alpha \rightarrow \beta$, defines that a value for $\alpha$ is sufficient to identify unique values for $\beta$ \cite{DBSBook}.
In this case, we say that $\alpha$ is the determinant and $\beta$ is the dependant. 
We say that $\beta$ is functionally determined by $\alpha$ if $\alpha \rightarrow \beta$.
It is logically implied that if $\alpha \rightarrow \beta$ and $\beta \rightarrow \gamma$, then $\gamma$ is functionally determined by $\alpha$.
If all attributes are functionally determined by $\alpha$, it is a super key\cite{DBSBook}.
The closure of functional dependency $\alpha$, denoted $\alpha^+$, describes all attributes that can be logically implied by functional dependencies having $\alpha$ as determinant\cite{DBSBook}. 

Boyce-Codd Normal Form (BCNF) eliminates all redundancy that can be discovered using functional dependencies\cite{DBSBook}. 
A relation is BCNF with respect to functional dependencies $F$ if, for all functional dependencies in $F^+$ of the form $\alpha \rightarrow \beta$ if one of the following holds:
$\alpha \rightarrow \beta$ is trivial ($\alpha \rightarrow \alpha$) or $\alpha$ is a super key for the relation.

Thus, if a relation $R$ is not in BCNF, there must be a functional dependency $\alpha \rightarrow \beta$ where $\alpha$ is not a super key. 
We can split such relation $R$ into two relations $R1(\alpha \cup \beta)$ and $R2(R-(\beta-\alpha))$ that are both in BCNF.

Third Normal Form (TNF) prevents partial and transitive dependencies\cite{MontayaNormalForms}.
That is, third normal form prevents functional dependencies of the form seen in equation \ref{eq:trans} 
and functional dependencies where non-prime attributes are functionally dependent on only a part of the candidate key. 
\begin{equation}\label{eq:trans}
    {\alpha} \rightarrow{\beta}, {\beta \rightarrow \gamma} \implies \alpha \rightarrow \gamma
\end{equation}

A relation is in TNF with respect to functional dependencies $F$ if, for all dependencies in $F^+$ of the form $\alpha \rightarrow \beta$, one of the following holds: 

$\alpha \rightarrow \beta$ is trivial, $\alpha$ is a super key, or each attribute $A \in \beta-\alpha$ is contained in a candidate key for $R$ \cite{DBSBook}.

Having established how one can evaluate a relational database design schema, we can proceed to discuss how a different model that can be used to represent the logic of a relational schema can make it easier to ensure that relations are of at least TNF.
