\subsection{Relational Databases}\label{relational_databases}
The group last year chose to implement the \texttt{WordCount} and \texttt{Fuseki} databases using PostgreSQL.
PostgreSQL is a relational database system \cite{knox2020}.


A relational database consists of several layers.
The lowest level is the physical layer which describes how the data is stored physically.
The purpose of the relational model is to abstract over the physical layer of the database.
This abstraction is known as the logical layer and allows database administrators to manage the physical storage without directly manipulating the physical data representation.
The logical layer describes what data is stored and the relationships between the data.
The highest of abstraction is the view layer which describes only part of the database. It exists to simplify the interaction with the system. Many views may exist for the same database.
\cite{DBSBook}

Compared to storing data on a regular file system, a database system provides many advantages including atomicity of operations, concurrent access to data, and lowered inconsistency and redundancy of data \cite{DBSBook}.
These advantages can be directly seen in the ACID properties that databases adhere to when performing a transaction \cite{DBSBook}.
\begin{itemize} \label{ACID}
    \item Atomicity: A transaction must either be fully completed or partial side-effects of a failed transaction must be undone.
    \item Consistency: A transaction in isolation must ensure values remain consistent after a transaction has been completed or terminated.
    \item Isolation: Transactions are unaware of other transactions being executed concurrently to avoid confusion.
    \item Durability: Changes caused by a committed transaction persist even in the event of system failures.
\end{itemize}

In the coming sections, we will describe how one can model and describe a relational database design, both mathematically and with a more graphic design.
After doing this, we will examine how operations on these models can be described using SQL and relational algebra.

\subsubsection*{Relational Model}
Relational database systems can be mathematically described using relations and sets, mapping a unique key to a tuple of information \cite{DBSBook}.
The values of the tuples contained within the relation can be described by the attributes of the relation and their corresponding domains \cite{KatjaFirstPP}. 
The relations are often described using a \textit{relational schema}, denoting the name and domains of the attributes.


Equation \ref{eq:relational_schema} shows an example of a relation describing books as tuples of three text fields (author\_name, title, and ISBN) and a positive integer (number\_of\_pages).
The relation also denotes a super key for the relation. A super key is one or more attributes that can uniquely identify a tuple in a given relation.
Any super key, where if just one attribute is removed it does no longer uniquely identify tuples, is called a \textit{candidate key}.
Attributes describing the super key are underlined.

\begin{equation} \label{eq:relational_schema}
    book(author\_name:text, title: text, number\_of\_pages:\mathbb{Z}^+, \underline{ISBN: text})
\end{equation}
Super keys can be defined as $t_1 \in r,\neq t_2 \in r \implies t_1.K \neq t_2.K$. 

That is, no two tuples $t_1, t_2$ from relation $r$ have same values for all super key attributes $K$. 
If the super key does not contain extraneous attributes, it is said to be \textit{minimal}. \cite{DBSBook}
We will use the term \textit{primary key} to denote a chosen super key of a relation. 
When describing a database, it is often necessary to specify how various data are connected. 
To do this, one can use \textit{foreign keys} to denote that tuples in $r_1$ are related to the tuples in $r_2$.


One could model the relationship between a book owner and a book using the relational schemas seen in equation \ref{eq:bookOwnerExample} and \ref{eq:relational_schema}.
There, primary keys from other relations are used to reference unique tuples. The $owns$ relation describes how relations $book$ and $book\_owner$ are connected. 
\begin{equation}\label{eq:bookOwnerExample}
    \begin{split}
        owns(\underline{owner\_id \rightarrow book\_owner}, \underline{ISBN \rightarrow book}), \\
        book\_owner(name:text,\underline{owner\_id:\mathbb{Z}^+})
    \end{split}
\end{equation}

\subsubsection*{Evaluating a Database Design}
When evaluating the design of relational database schemas, we want to avoid redundant data duplication, loss by decomposition, change of dependencies \cite{DBSBook}.
One approach to ensure this, is to use normalization theory \cite{DBSBook}. This approach examines the functional dependencies of the relations and evaluate them based on their \textit{normal form}.
When a relation is of a normal form, it adheres to certain design criteria. These criteria are defined using \textit{functional dependencies}.

A functional dependency describes the relationship between two sets of attributes. 
Functional dependency between $\alpha$ and $\beta$ written as $\alpha \rightarrow \beta$ defines that a value for $\alpha$ is sufficient to identify unique values for $\beta$ \cite{DBSBook}.
In this case, we say that $\alpha$ is the determinant and $\beta$ is the dependant. 
We say that $\beta$ is functionally determined by $\alpha$ if $\alpha \rightarrow \beta$.
It is logically implied that if $\alpha \rightarrow \beta$ and $\beta \rightarrow \gamma$ then $\gamma$ is functionally determined by $\alpha$.
If all attributes are functionally determined by $alpha$ it is a super key \cite{DBSBook}.
The closure of functional dependency $\alpha$ denoted $\alpha^+$ describes all attributes that can be logically implied by functional dependencies having $\alpha$ as determinant \cite{DBSBook}. 

Boyce-Codd Normal Form (BCFN) eliminates all redundancy that can be discovered using functional dependencies \cite{DBSBook}. 
A relation is BCNF with respect to $F$ if, for all functional dependencies in $F^+$ of the form $\alpha \rightarrow \beta$ if one of the following holds:
$\alpha \rightarrow \beta$ is trivial ($\alpha \rightarrow \alpha$) or $\alpha$ is a super key for the relation.

Thus, if a relation $R$ is not in BCNF, there must be a functional dependency $\alpha \rightarrow \beta$ where $\alpha$ is not a super key. 
We can split such relation $R$ into two relations $R1(\alpha \cup \beta)$ and $R2(R-(\beta-\alpha))$ that are both in BCNF.

Third Normal Form (TNF) prevents partial and transitive dependencies \cite{MontayaNormalForms}.
A relation is in TNF with respect to functional dependencies $F$ is for all dependencies in $F^+$ of the form $\alpha \rightarrow \beta$ if one of the following holds: 
$\alpha \rightarrow \beta$ is trivial, $\alpha$ is a super key, or each attribute $A \in \beta-\alpha$ is contained in a candidate key for $R$ \cite{DBSBook}.

Having established how one can evaluate a relational database design schema we can proceed to discuss how a different model that can be used to represent the logic of a relational schema can make it easier to ensure that relations are of at least TNF.


