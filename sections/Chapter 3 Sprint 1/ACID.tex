\subsubsection*{ACID Principles}\label{sec:ACID}

A relational database consists of several layers, each layer abstracting over concepts in the database.
The lowest level is the physical layer, which describes how the data are stored physically.
The purpose of the relational model is to abstract over the physical layer of the database\cite{DBSBook}.

This abstraction is known as the logical layer and allows database administrators to manage the physical storage without directly manipulating the physical data representation.
The logical layer describes what are stored and the relationships between the data\cite{DBSBook}.
The highest of abstraction is the view layer, which describes only part of the database\cite{DBSBook}.
It exists to simplify the interaction with the system. Many views may exist for the same database.
\cite{DBSBook}
We have seen that SQL can be used to describe operations performed on this abstraction, both to define and to query data.

Compared to storing data using a regular file system, a database system provides many advantages including atomicity of operations, concurrent access to data, and lowered inconsistency and redundancy of data\cite{DBSBook}.
These advantages can be directly seen in the ACID properties that databases adhere to when performing a transaction\cite{DBSBook}.
\begin{itemize} \label{ACID}
    \item Atomicity: A transaction must either be fully completed or partial side-effects of a failed transaction must be undone.
    \item Consistency: A transaction in isolation must ensure values remain consistent after a transaction has been completed or terminated.
    \item Isolation: Transactions are unaware of other transactions being executed concurrently to avoid confusion.
    \item Durability: Changes caused by a committed transaction persist even in the event of system failures.
\end{itemize}
