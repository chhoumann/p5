\subsection{Relational Databases}
The group last year chose to implement the \texttt{WordCount} (Section \ref{sec:Sprint2And3}) and \texttt{Fuseki} databases using PostgreSQL.
Postgres is a relational database system \cite{knox2020}.
Relational database systems can be mathematically described using relations, mapping a unique key to a tuple of information \cite[Chapter~2.3]{DBSBook}.
The values of the tuples contained within the relation can be described by the attributes of the relation and their corresponding domains \cite{KatjaFirstPP}. 
The relations are often described using a \textit{relational schema}, denoting the name and domains of the attributes.
Equation \ref{eq:relational_schema} shows an example of a relation describing books as tuples containing three text fields and a positive integer.
The relation also denotes a super key for the relation. A super key is one or more attributes that can uniquely identify a tuple in a given relation. Attributes describing the superkey are underlined.

\begin{equation} \label{eq:relational_schema}
    book(author\_name:text, title: text, number\_of\_pages:\mathbb{Z}^+, \underline{ISBN: text})
\end{equation}
Superkeys can be defined as $t_1 \in r,\neq t_2 \in r \implies t_1.K \neq t_2.K$. 

That is, no two tuples $t_1, t_2$ from relation $r$ have same values for all superkey attributes $K$. 
If the superkey does not contain extraneous attributes it is said to be \textit{minimal}. \cite[Chapter 2.3]{DBSBook}
We will use the term \textit{primary key} to denote a chosen superkey of a relation. 
When describing a database, it is often necessary to describe how various data is connected. 
To do this, one can use \textit{foreign keys} to denote that tuples of $r_1$ is related to $r_2$.
One could model the relationship between a book owner and a book using the relational schemas seen in Equation \ref{eq:bookOwnerExample} and \ref{eq:relational_schema}.
There, primary keys from other relations are used to reference unique tuples. The $owns$ relation describes how $book$ and $book\_owner$ are connected. 
\begin{equation}\label{eq:bookOwnerExample}
    \begin{split}
        owns(\underline{owner\_id \rightarrow book\_owner}, \underline{ISBN \rightarrow book}), \\
        book\_owner(name:text,\underline{owner\_id:\mathbb{Z}^+})
    \end{split}
\end{equation}

Having established the fundamentals of relational databases, we will now describe the operations that can be performed on relations. 

\subsubsection*{Relational algebra}
Relational algebra is a set of unary and binary operations on relations that produces new relations \cite[Chapter 2.6]{DBSBook}.

\subsubsection*{Database design}



A relational database is a collection of tables, known as relations. Relational databases are used to provide access to related data points.
These relations consist of rows, known as tuples. Each tuple represents an entity with specific attributes represented by the columns of the given relation.
In the context of an enterprise company, one could for example model an employee and a manager, and the relationship between them.
The goal of the designing such relations is to represent entities as intuitively as possible, making it easy to establish relationships between them. 

To distinguish between tuples, keys are used to ensure that tuples in a relation can be uniquely identified.
Foreign keys are used for modelling relationships between relations where foreign key in one relation represents the primary key of another relation.

Data from a relational database can be queried using relational query languages such as SQL.
Executing a query instructs the database system to perform a set of operations to compute a desired result, known as transactions.

These transactions encapsulate several atomic operations and thus the transactions themselves are not inherently atomic which can cause data inconsistencies. 
To prevent this from happening, a relational database must ensure four crucial properties:
\begin{itemize}
    \item Atomicity: A transaction must either be fully completed or partial side-effects of a failed transaction must be undone.
    \item Consistency: A transaction in isolation must ensure values remain consistent after a transaction has been completed or terminated.
    \item Isolation: Transactions are unaware of other transactions being executed concurrently to avoid confusion.
    \item Durability: Changes caused by a committed transaction persist even in the event of system failures.
\end{itemize}

A relational database consists of several layers.
The lowest level is the physical layer which describes how the data are stored physically.
The purpose of the relational model is abstract over the physical layer of the database.

This abstraction is known as the logical layer and allows database administrators to manage the physical storage without directly manipulating the physical data representation.
The logical layer describes what data are stored and the relationships between the data.

The highest of abstraction is the view layer which describes only part of the database. It exits to simplify the interaction with the system. Many views may exist for the same database.

Entity-relation diagrams (ER diagrams) can be used to express the relationships in a relational database in a concise and graphical manner \cite{DBSBook} \cite{OracleRDBMS}.