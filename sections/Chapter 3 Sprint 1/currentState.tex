\section{The current state of \knox{}}
\subsection{Server Distribution and Their Specifications}
The \knox{} server system consists of six servers, each running the Ubuntu 18.04 LTS operating system.
The servers each have 4 GiB of RAM, 2 vCPUs, and about 100 GB of hard drive space. 

Four of these servers have no public ports open, and serve as containers for running the various applications of the \knox{} pipeline.
These servers will be referred to as \textit{node servers} or \textit{server nodes}.
The remaining two servers have port 80 and 443 open to public. These nodes will be referred to as \textit{master nodes}.
Currently, one of the master nodes run Grafana\cite{GRAFANA}, Prometheus\cite{Prometheus}, Apache\cite{ApacheJena}, and Docker\cite{Docker_IBM}.
The other master node only has a single running program, NMap\cite{NMap}.
The Docker instance runs containers based on images of Portainer, Wiki.js, as well as a PostgreSQL database.
Many of the node servers run applications written in Java and Python, as well as databases for these applications. 
No firewalls are set in place between the servers.

Having established an overview of the programs and running on the servers, we will now take a closer look at the databases of the third pipeline layer.

\subsection{Relational databases}
The group last year chose to implement the \texttt{WordCount} (see Section \ref{sec:Sprint2And3}) and \texttt{Fuseki} databases using PostgreSQL.
Postgres is a relational database system \cite{knox2020}.


A relational database consists of several layers.
The lowest level is the physical layer which describes how the data are stored physically.
The purpose of the relational model is to abstract over the physical layer of the database.
This abstraction is known as the logical layer and allows database administrators to manage the physical storage without directly manipulating the physical data representation.
The logical layer describes what data are stored and the relationships between the data.
The highest of abstraction is the view layer which describes only part of the database. It exists to simplify the interaction with the system. Many views may exist for the same database.
\cite[Chapter 1.3]{DBSBook}

Compared to storing data on a regular file system, a database system provides many advantages including atomicity of operations, concurrent access to data, and lowered inconsistency and redundancy of data \cite[Chapter 1.2]{DBSBook}.
These advantages can be directly seen in the ACID properties that databases adhere to when performing a transaction (see Section \ref{sec:SQL}) \cite[Chapter~17]{DBSBook}.
\begin{itemize} \label{ACID}
    \item Atomicity: A transaction must either be fully completed or partial side-effects of a failed transaction must be undone.
    \item Consistency: A transaction in isolation must ensure values remain consistent after a transaction has been completed or terminated.
    \item Isolation: Transactions are unaware of other transactions being executed concurrently to avoid confusion.
    \item Durability: Changes caused by a committed transaction persist even in the event of system failures.
\end{itemize}

In the coming sections, we will describe how one can model and describe a relational database design, both mathematically and with a more graphic design.
After doing this, we will examine how operations on these models can be described using SQL (Section~\ref{sec:SQL}) and relational algebra (Section~\ref{sec:relationalAlgebra}).

\subsubsection*{Relational model}
Relational database systems can be mathematically described using relations and sets, mapping a unique key to a tuple of information \cite[Chapter~2.3]{DBSBook}.
The values of the tuples contained within the relation can be described by the attributes of the relation and their corresponding domains \cite{KatjaFirstPP}. 
The relations are often described using a \textit{relational schema}, denoting the name and domains of the attributes.


Equation \ref{eq:relational_schema} shows an example of a relation describing books as quadruples of three text fields (author\_name, title, and ISBN) and a positive integer (number\_of\_pages).
The relation also denotes a super key for the relation. A super key is one or more attributes that can uniquely identify a tuple in a given relation. Attributes describing the super key are underlined.

\begin{equation} \label{eq:relational_schema}
    book(author\_name:text, title: text, number\_of\_pages:\mathbb{Z}^+, \underline{ISBN: text})
\end{equation}
Super keys can be defined as $t_1 \in r,\neq t_2 \in r \implies t_1.K \neq t_2.K$. 

That is, no two tuples $t_1, t_2$ from relation $r$ have same values for all super key attributes $K$. 
If the super key does not contain extraneous attributes, it is said to be \textit{minimal}. \cite[Chapter 2.3]{DBSBook}
We will use the term \textit{primary key} to denote a chosen super key of a relation. 
When describing a database, it is often necessary to specify how various data are connected. 
To do this, one can use \textit{foreign keys} to denote that tuples in $r_1$ are related to the tuples in $r_2$.


One could model the relationship between a book owner and a book using the relational schemas seen in equation \ref{eq:bookOwnerExample} and \ref{eq:relational_schema}.
There, primary keys from other relations are used to reference unique tuples. The $owns$ relation describes how relations $book$ and $book\_owner$ are connected. 
\begin{equation}\label{eq:bookOwnerExample}
    \begin{split}
        owns(\underline{owner\_id \rightarrow book\_owner}, \underline{ISBN \rightarrow book}), \\
        book\_owner(name:text,\underline{owner\_id:\mathbb{Z}^+})
    \end{split}
\end{equation}

Instead of describing the data structures of the database in these relations, one can use a different model that represents the logic of the relational schemas.

\subsubsection{Entity relation model}\label{sec:EntityRelationModel}
The entity relation model is a graphical representation of how database relations, are structured.
This is done in a \textit{E-R diagram}, and is commonly used facilitate database design from specifications of enterprise schemas \cite*[Chapter 6.2]{DBSBook}.
However, in the E-R model, we differentiate between \textit{relationships} connecting relations and \textit{entity sets} representing domain elements. 
The entity sets are represented graphically with a rectangle and relationships between the entity sets with a diamond. \cite[Chapter 6.2]{DBSBook}
The attribute associated with the entity sets can be modelled using ovals \cite{KatjaFirstPP}, however other alternatives exists \footnotetext{Alternatives diagram styles are presented in Chapter 6.10 of \cite{DBSBook}}
Similar to the mathematical approach, the entities (tuples) must be uniquely identified by one or more attributes. This is denoted by underlining the attributes. 
The E-R model have notations that denotes the participations each entity have in a connecting relationship \cite[Chapter 6.4]{DBSBook}.
In this project, we will use \text{min-max} notation; this notation denotes the minimum and maximum amount of entities participating in a relationship. 

\begin{figure}[htp]
    \centering
    \includegraphics[scale=0.5]{Images/book_example_w_cardinality.png}
    \caption{E-R diagram of the book and book owner example.}
    \label{fig:ER_Book_Example}
\end{figure}

Figure \ref{fig:ER_Book_Example} shows the $book$, $owns$, and $book_owner$ relations from equation \ref{eq:bookOwnerExample} and \ref{eq:relational_schema}.
In figure \ref{fig:ER_Book_Example} we also see the  the participation cardinalities of the two entity sets. 
The relationship from $book\_owner$ to $owns$ is a one-to-many relationship. That is, each book owner entity must own at least one book. Thus the relationship has full participation. 
\begin{figure}[h]
    \centering
    \includegraphics[scale=0.5]{Images/cardinalities.png}
    \caption{Participation ratios for ER relationships}
    \label{fig:ERDiagram_Cardinality}
\end{figure}
The relationship from $book$ to $owns$ is zero-to-many. That is, the diagram represents books that no one owns, and that a book can be owned by multiple book owners.
This relationship does not have full participation.
Participation connections can be seen on figure \ref{fig:ERDiagram_Cardinality}.


Other forms of entity sets exits. For instance, a weak entity set, denoted by a an oval with doubled edges, is an entity sets whose existence is based on another entity set.
A weak entity set is identified by its \textit{identifying entity set}'s primary key along with extra attributes. 
The relationship between weak entity sets and its identifying set is always many-to-one with full participation of the weak entity set.


Since the relational model is described using sets, we can also describe operations that can be performed on the relations.
Therefore, it is useful to be able to convert an E-R model into an equivalent relational model, where domains of attributes can be well defined and operations on the entity sets described mathematically.
\subsubsection*{Converting E-R models to relations}



\subsubsection{Relational algebra}\label{sec:relationalAlgebra}
Relational algebra describes a set of unary and binary operations on relations that produce new relations.
These operators form the foundations of data manipulation languages (see Section \ref{sec:SQL}) which can be used to define database operations \cite[Chapter 6.2]{DBSBook}.
We will describe some of the operators of relational algebra, and use them to describe implemented queries later.





\begin{table}[h]
    \centering
    \begin{tabular}{|lll|}
    \hline 
    \multicolumn{1}{|l|}{\textbf{Operator}}          & \multicolumn{1}{l|}{\textbf{Example}}   & \multicolumn{1}{l|}{\textbf{Is unary}}      \\ \hline
    \multicolumn{1}{|l|}{Select}                     & \multicolumn{1}{l|}{$\sigma_{predicate}(R)$}             & \multicolumn{1}{l|}{$\checkmark$}           \\ \hline
    \multicolumn{1}{|l|}{Projection}                 & \multicolumn{1}{l|}{$\pi_{A_1, A2,...,A_n}(R)$}             & \multicolumn{1}{l|}{$\checkmark$}           \\ \hline
    \multicolumn{1}{|l|}{Join}                       & \multicolumn{1}{l|}{$r_1 \Join r_2$}             & \multicolumn{1}{l|}{$\times$}           \\ \hline
    \multicolumn{1}{|l|}{Theta Join}                 & \multicolumn{1}{l|}{$r_1 \Join_\Theta r_2$}             & \multicolumn{1}{l|}{$\times$}           \\ \hline
    \multicolumn{1}{|l|}{Cartesian product}          & \multicolumn{1}{l|}{$r_1\times r_2$}              & \multicolumn{1}{l|}{$\times$}            \\ \hline
    \end{tabular}
    \caption{Table of operators}
    \label{Relational algebra operators}
\end{table}



\subsubsection{SQL}\label{sec:SQL}
-> What is SQL
-> What is DDL
-> What is DML
   --> How is it similar to relational algebra?





\subsubsection*{Evaluating a database design}

Data from a relational database can be queried using relational query languages such as SQL.
Executing a query instructs the database system to perform a set of operations to compute a desired result. Collectively, the resulting set of operations is known as a transaction.

\subsection{The current \knox{} databases}
\subsubsection*{Relational databases}
To summarize, the previous group set up a PostgreSQL database on \texttt{node02}.

Figure \ref{olddatabase} shows an ER diagram of the database as we received it from the previous group.

\begin{figure}[h]
    \centering
    \includegraphics[width=\linewidth]{Images/old_db_er_diagram.png}
    \caption{ER diagram of the relational database from last year.}
    \label{olddatabase}
\end{figure}

\subsubsection*{Fuseki database}
The previous group set Apache Jena Fuseki on \texttt{node01}. They chose to use HDT format, claiming that it is faster for querying than TDP\cite{knox2020}.

The setup is used to store the knowledge graph used in \knox{}.

\subsection{Database endpoints and implemented applications}

As previously mentioned, \knox{} was initially started in 2020.
The previous group chose to develop API endpoints using Java.
They set up a simple database system using the Apache Jena framework in combination with the Fuseki package.

The system has two functionalities. 
First, it can store knowledge graphs in HDT and convert triples to HDT.
Second, it contains a prototype for counting the number of words in an article.
There is no API for reading this data, only for writing. 
In addition, at least one of other layers are directly connected to the database.

Overall, not much code was written, and its quality is unknown due to a lack of testing and documentation.
The Java language convention was also not followed \cite{java_convention}, making it difficult to comprehend.
It appears as though no structure was established, making it difficult to navigate the code. 
Furthermore, the database is set up such that it must be restarted every time new data is written to it - otherwise, the data cannot be fetched by the other layers.

Based on this, we have discussed what work to do and in what order.
After having read and understood the current code, we will have to start by cleaning up the inconsistencies and structure.
We will then have to document what the code does and write tests for it.
We must also address the unfortunate database implementation such that constantly restarting is no longer required.
Finally, we need to decouple the other layers such that they are no longer directly connected to the database.

To solve these issues, it was decided that the best approach was to discard all current implementations. 
This decision was based on several factors such as unfamiliarity with the environment and the previously mentioned poor structure. 
Discarding the current implementation also allows us to build the layer using C\#, a programming language that we are more familiar with which is also taught in the future semesters. 
This will make it easier for the following groups to continue and not have to learn a new language.

While doing so, we will follow a proper structure and write tests and documentation along the way. 
Moreover, it would make the system more accessible to future students as C\# is taught on the 3rd semester.