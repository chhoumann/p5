\section{The Current State of \knox{}} \label{currentState}
\subsection{Server Distribution and Their Specifications}
The \knox{} server system consists of six servers, each running the Ubuntu 18.04 LTS operating system.
The servers each have 4 GiB of RAM, 2 vCPUs, and about 100 GB of hard drive space. 

Four of these servers have no public ports open, and serve as containers for running the various applications of the \knox{} pipeline.
These servers will be referred to as \textit{node servers} or \textit{server nodes}.
The remaining two servers have port 80 and 443 open to public. These nodes will be referred to as \textit{master nodes}.
Currently, one of the master nodes run Grafana\cite{GRAFANA}, Prometheus\cite{Prometheus}, Apache\cite{ApacheJena}, and Docker\cite{Docker_IBM}.
The other master node only has a single running program, NMap\cite{NMap}.
The Docker instance runs containers based on images of Portainer, Wiki.js, as well as a PostgreSQL database.
Many of the node servers run applications written in Java and Python, as well as databases for these applications. 
No firewalls are set in place between the servers.

Having established an overview of the programs and running on the servers, we will now take a closer look at the databases of the third pipeline layer.

\subsection{Relational databases}
The group last year chose to implement the \texttt{WordCount} (see Section \ref{sec:Sprint2And3}) and \texttt{Fuseki} databases using PostgreSQL.
Postgres is a relational database system \cite{knox2020}.


A relational database consists of several layers.
The lowest level is the physical layer which describes how the data are stored physically.
The purpose of the relational model is to abstract over the physical layer of the database.
This abstraction is known as the logical layer and allows database administrators to manage the physical storage without directly manipulating the physical data representation.
The logical layer describes what data are stored and the relationships between the data.
The highest of abstraction is the view layer which describes only part of the database. It exists to simplify the interaction with the system. Many views may exist for the same database.
\cite[Chapter 1.3]{DBSBook}

Compared to storing data on a regular file system, a database system provides many advantages including atomicity of operations, concurrent access to data, and lowered inconsistency and redundancy of data \cite[Chapter 1.2]{DBSBook}.
These advantages can be directly seen in the ACID properties that databases adhere to when performing a transaction (see Section \ref{sec:SQL}) \cite[Chapter~17]{DBSBook}.
\begin{itemize} \label{ACID}
    \item Atomicity: A transaction must either be fully completed or partial side-effects of a failed transaction must be undone.
    \item Consistency: A transaction in isolation must ensure values remain consistent after a transaction has been completed or terminated.
    \item Isolation: Transactions are unaware of other transactions being executed concurrently to avoid confusion.
    \item Durability: Changes caused by a committed transaction persist even in the event of system failures.
\end{itemize}

In the coming sections, we will describe how one can model and describe a relational database design, both mathematically and with a more graphic design.
After doing this, we will examine how operations on these models can be described using SQL (Section~\ref{sec:SQL}) and relational algebra (Section~\ref{sec:relationalAlgebra}).

\subsubsection*{Relational model}
Relational database systems can be mathematically described using relations and sets, mapping a unique key to a tuple of information \cite[Chapter~2.3]{DBSBook}.
The values of the tuples contained within the relation can be described by the attributes of the relation and their corresponding domains \cite{KatjaFirstPP}. 
The relations are often described using a \textit{relational schema}, denoting the name and domains of the attributes.


Equation \ref{eq:relational_schema} shows an example of a relation describing books as quadruples of three text fields (author\_name, title, and ISBN) and a positive integer (number\_of\_pages).
The relation also denotes a super key for the relation. A super key is one or more attributes that can uniquely identify a tuple in a given relation. Attributes describing the super key are underlined.

\begin{equation} \label{eq:relational_schema}
    book(author\_name:text, title: text, number\_of\_pages:\mathbb{Z}^+, \underline{ISBN: text})
\end{equation}
Super keys can be defined as $t_1 \in r,\neq t_2 \in r \implies t_1.K \neq t_2.K$. 

That is, no two tuples $t_1, t_2$ from relation $r$ have same values for all super key attributes $K$. 
If the super key does not contain extraneous attributes, it is said to be \textit{minimal}. \cite[Chapter 2.3]{DBSBook}
We will use the term \textit{primary key} to denote a chosen super key of a relation. 
When describing a database, it is often necessary to specify how various data are connected. 
To do this, one can use \textit{foreign keys} to denote that tuples in $r_1$ are related to the tuples in $r_2$.


One could model the relationship between a book owner and a book using the relational schemas seen in equation \ref{eq:bookOwnerExample} and \ref{eq:relational_schema}.
There, primary keys from other relations are used to reference unique tuples. The $owns$ relation describes how relations $book$ and $book\_owner$ are connected. 
\begin{equation}\label{eq:bookOwnerExample}
    \begin{split}
        owns(\underline{owner\_id \rightarrow book\_owner}, \underline{ISBN \rightarrow book}), \\
        book\_owner(name:text,\underline{owner\_id:\mathbb{Z}^+})
    \end{split}
\end{equation}

Instead of describing the data structures of the database in these relations, one can use a different model that represents the logic of the relational schemas.

\subsubsection{Entity relation model}\label{sec:EntityRelationModel}
The entity relation model is a graphical representation of how database relations, are structured.
This is done in a \textit{E-R diagram}, and is commonly used facilitate database design from specifications of enterprise schemas \cite*[Chapter 6.2]{DBSBook}.
However, in the E-R model, we differentiate between \textit{relationships} connecting relations and \textit{entity sets} representing domain elements. 
The entity sets are represented graphically with a rectangle and relationships between the entity sets with a diamond. \cite[Chapter 6.2]{DBSBook}
The attribute associated with the entity sets can be modelled using ovals \cite{KatjaFirstPP}, however other alternatives exists \footnotetext{Alternatives diagram styles are presented in Chapter 6.10 of \cite{DBSBook}}
Similar to the mathematical approach, the entities (tuples) must be uniquely identified by one or more attributes. This is denoted by underlining the attributes. 
The E-R model have notations that denotes the participations each entity have in a connecting relationship \cite[Chapter 6.4]{DBSBook}.
In this project, we will use \text{min-max} notation; this notation denotes the minimum and maximum amount of entities participating in a relationship. 

\begin{figure}[htp]
    \centering
    \includegraphics[scale=0.5]{Images/book_example_w_cardinality.png}
    \caption{E-R diagram of the book and book owner example.}
    \label{fig:ER_Book_Example}
\end{figure}

Figure \ref{fig:ER_Book_Example} shows the $book$, $owns$, and $book_owner$ relations from equation \ref{eq:bookOwnerExample} and \ref{eq:relational_schema}.
In figure \ref{fig:ER_Book_Example} we also see the  the participation cardinalities of the two entity sets. 
The relationship from $book\_owner$ to $owns$ is a one-to-many relationship. That is, each book owner entity must own at least one book. Thus the relationship has full participation. 
\begin{figure}[h]
    \centering
    \includegraphics[scale=0.5]{Images/cardinalities.png}
    \caption{Participation ratios for ER relationships}
    \label{fig:ERDiagram_Cardinality}
\end{figure}
The relationship from $book$ to $owns$ is zero-to-many. That is, the diagram represents books that no one owns, and that a book can be owned by multiple book owners.
This relationship does not have full participation.
Participation connections can be seen on figure \ref{fig:ERDiagram_Cardinality}.


Other forms of entity sets exits. For instance, a weak entity set, denoted by a an oval with doubled edges, is an entity sets whose existence is based on another entity set.
A weak entity set is identified by its \textit{identifying entity set}'s primary key along with extra attributes. 
The relationship between weak entity sets and its identifying set is always many-to-one with full participation of the weak entity set.


Since the relational model is described using sets, we can also describe operations that can be performed on the relations.
Therefore, it is useful to be able to convert an E-R model into an equivalent relational model, where domains of attributes can be well defined and operations on the entity sets described mathematically.
\subsubsection*{Converting E-R models to relations}



\subsubsection{Relational algebra}\label{sec:relationalAlgebra}
Relational algebra describes a set of unary and binary operations on relations that produce new relations.
These operators form the foundations of data manipulation languages (see Section \ref{sec:SQL}) which can be used to define database operations \cite[Chapter 6.2]{DBSBook}.
We will describe some of the operators of relational algebra, and use them to describe implemented queries later.





\begin{table}[h]
    \centering
    \begin{tabular}{|lll|}
    \hline 
    \multicolumn{1}{|l|}{\textbf{Operator}}          & \multicolumn{1}{l|}{\textbf{Example}}   & \multicolumn{1}{l|}{\textbf{Is unary}}      \\ \hline
    \multicolumn{1}{|l|}{Select}                     & \multicolumn{1}{l|}{$\sigma_{predicate}(R)$}             & \multicolumn{1}{l|}{$\checkmark$}           \\ \hline
    \multicolumn{1}{|l|}{Projection}                 & \multicolumn{1}{l|}{$\pi_{A_1, A2,...,A_n}(R)$}             & \multicolumn{1}{l|}{$\checkmark$}           \\ \hline
    \multicolumn{1}{|l|}{Join}                       & \multicolumn{1}{l|}{$r_1 \Join r_2$}             & \multicolumn{1}{l|}{$\times$}           \\ \hline
    \multicolumn{1}{|l|}{Theta Join}                 & \multicolumn{1}{l|}{$r_1 \Join_\Theta r_2$}             & \multicolumn{1}{l|}{$\times$}           \\ \hline
    \multicolumn{1}{|l|}{Cartesian product}          & \multicolumn{1}{l|}{$r_1\times r_2$}              & \multicolumn{1}{l|}{$\times$}            \\ \hline
    \end{tabular}
    \caption{Table of operators}
    \label{Relational algebra operators}
\end{table}



\subsubsection{SQL}\label{sec:SQL}
-> What is SQL
-> What is DDL
-> What is DML
   --> How is it similar to relational algebra?





\subsubsection*{Evaluating a database design}

Data from a relational database can be queried using relational query languages such as SQL.
Executing a query instructs the database system to perform a set of operations to compute a desired result. Collectively, the resulting set of operations is known as a transaction.

\subsubsection{Entity Relationship Model}\label{sec:EntityRelationModel}
The entity relationship model is a graphical representation of how database relations are structured.
This is done in a \textit{entity relation} (ER) diagram}, and is commonly used facilitate database design from specifications of enterprise schemas \cite{DBSBook}.
In the ER model, we differentiate between \textit{entity sets} representing domain elements and \textit{relationships sets} representing the connections between the domain elements. 
The entity sets are represented graphically with a rectangle and relationship sets connecting the entity sets with a diamond \cite{DBSBook}.
The attributes associated with the entity- and relationship sets can be modelled using ovals \cite{KatjaFirstPP}, however other alternatives exists.\footnote{Our style of choice is presented in Chapter 6.10 of \citetitle{DBSBook} \textit{\citefield[]{DBSBook}[]{edition}}}
Similar to the relational model, the entities (tuples) must be uniquely identified by one or more attributes. This is denoted by underlining the attributes. 

The E-R model have notations that denotes the participations each entity have in a connecting relationship \cite{DBSBook}.
In this project, we will use \text{min-max} notation; this notation denotes the minimum and maximum amount of entities participating in a relationship. 

\begin{figure}[htp]
    \centering
    \includegraphics[scale=0.5]{Images/book_example_w_cardinality.png}
    \caption{E-R diagram of the book and book owner example.}
    \label{fig:ER_Book_Example}
\end{figure}

Figure \ref{fig:ER_Book_Example} shows an E-R diagram equivalent to the $book$, $owns$, and $book_owner$ relations from equation \ref{eq:bookOwnerExample} and \ref{eq:relational_schema}.
In figure \ref{fig:ER_Book_Example} we also see the  the participation cardinalities of the two entity sets. 
The relationship from $book\_owner$ to $owns$ is one of total participation. That is, each book owner entity must own at least one book.
\begin{figure}[h]
    \centering
    \includegraphics[scale=0.5]{Images/cardinalities.png}
    \caption{Participation ratios for ER relationships}
    \label{fig:ERDiagram_Cardinality}
\end{figure}
The participation from $book$ to $owns$ is partial. Thus the model represents that books can be in the database, even if no one owns it, that a book can be owned by multiple book owners.
Participation connections can be seen on figure \ref{fig:ERDiagram_Cardinality}.


Other forms of entity sets exits. For instance, a weak entity set, denoted by a an oval with doubled edges, is an entity sets whose existence is based on another entity set. Relationship sets connecting a weak entity set to its' identifying entity set will not have any attributes.
A weak entity set is identified by its \textit{identifying entity set}'s primary key along with extra attributes. 
The relationship between weak entity sets and its identifying set is always many-to-one with total  participation of the weak entity set.

Since the relational model is described using relations, we can also describe operations that can be performed on the relations.
Therefore, it is useful to be able to convert an E-R model into an equivalent relational model, where domains of attributes can be well defined and operations on the entity sets described mathematically.

\subsubsection*{Converting E-R Models to Relations}
\textbf{Strong entity sets}\\
Strong entity sets are converted by taking each attribute of the entity set, and a relation with corresponding attributes. 
The primary key of the entity set is chosen as primary key for the relation \cite{DBSBook}.\\
As an example of converting a strong entity we can use the book owner example from figure \ref{fig:ER_Book_Example}. In this example \texttt{Book} is a strong entity. The relation created from the book entity can be seen in equation \ref*{eq:bookConversion}. The attributes from the entity are added to the relation and the primary key of the book entity, \texttt{ISBN}, is used as the primary key for the relation.

\begin{equation}\label{eq:bookConversion}
    \begin{split}
        Book(\underline{ISBN : text} , title : text , author\_name : text , number\_of\_pages)
    \end{split}
\end{equation}
\\
% \begin{equation}\label{eq:bookOwnerExample}
%     \begin{split}
%         owns(\underline{owner\_id \rightarrow book\_owner}, \underline{ISBN \rightarrow book}), \\
%         book\_owner(name:text,\underline{owner\_id:\mathbb{Z}^+})
%     \end{split}
% \end{equation}
\textbf{Many-to-many relationships}\\
When converting many-to-many relationships to a relation, one create a single relation with primary key attributes from the participating entity sets \cite{DBSBook}. These keys are used as foreign key to reference the converted entities participating in the relationship. 
The remaining attributes of the entity set is similarly mapped to the relation.\\
\textbf{Many-to-one relationships}\\
If we have a many-to-one relationship between two sets we can combine the relationship set and the relation created from the entity set on the 'many' side into a single relation. This relation will use the primary key of the entity set as its primary key, and have a foreign referencing the relation created from the 'one' side of the relationship.\\
\textbf{One-to-one relationships}\\
When converting one-to-one relationships with total participation, we simply create a union of the attributes of the participating entity sets and the relationship set \cite{DBSBook}. If there is not total participation, two relations are created from the entity set, and a foreign key is placed on one of the sets.\\
\textbf{Weak entity sets}\\
When representing weak entity sets, one creates a relation containing the attributes of the weak entity set as well as the attributes of the identifying set's primary key.
The primary key of the identifying relation will also serve as a foreign key to the identifying relation. \cite{DBSBook}.
The relationship set is simply ignored, as the relationship set will not have any attributes \cite{DBSBook}. 

Having established how to convert an ER diagram into an equivalent relational model, we can now describe the operations for the relational model.

\subsubsection{Relational Algebra}\label{sec:relationalAlgebra}
Relational algebra describes a set of unary and binary operations on relations.
The operations are used to define new relations --- often the tuples within the set will satisfy defined predicates.
The relational operators form the foundations of data manipulation languages (see section \ref{sec:SQL}) which can be used to define database operations\cite{DBSBook}.
A few of these operations can be seen in table \ref{Relational algebra operators}.


\begin{table}[h]
    \centering
    \begin{tabular}{|ll|}
    \hline 
    \multicolumn{1}{|l|}{\textbf{Operator}}          & \multicolumn{1}{l|}{\textbf{Example}}   \\ \hline
    \multicolumn{1}{|l|}{Select}                     & \multicolumn{1}{l|}{$\sigma_{predicate}(R)$}            \\ \hline
    \multicolumn{1}{|l|}{Projection}                 & \multicolumn{1}{l|}{$\pi_{A_1, A2,...,A_n}(R)$}           \\ \hline
    \multicolumn{1}{|l|}{Join}                 & \multicolumn{1}{l|}{$r_1 \Join_\Theta r_2$}             \\ \hline
    \multicolumn{1}{|l|}{Cartesian product}          & \multicolumn{1}{l|}{$r_1\times r_2$}              \\ \hline
    \end{tabular}
    \caption{Table of operators}
    \label{Relational algebra operators}
\end{table}

We will describe some operators of relational algebra, and use them to describe implemented queries later.
Some common operators of in relational algebra can be seen in table \ref{Relational algebra operators}.
\subsubsection*{Select}
The select operator defines a set, $R_{result}\subseteq R$ where all tuples $t \in R_{result}$ satisfies a given predicate\cite{DBSBook}.
That is, $\forall t \in R_{result} \vDash predicate$.
As an example, we can produce a relation containing all books with the title "Database System Concepts":
$$\sigma_{author\_name = "Database System Concepts"}(books)$$
\subsubsection*{Projection}
The projection operator specifies a relation containing only a subset of the attributes of the operand relation\cite{DBSBook}.
That is, attributes $A_1, ..., A_n$ of relation $R_{result}$ will be a subset or equal to the attributes of the operand.
We can use the projection in combination with the select operator, to produce a relation containing only the ISBN and title of books titled "Database System Concepts":
$$\pi_{ISBN,title} (\sigma_{author\_name = "Database\ System\ Concepts"}(books))$$
\subsubsection*{Cartesian product}
The Cartesian product of two relations $R_1$ and $R_2$ produces a set containing concatenated tuples of the two operands.
That is, taking the attributes of the two sets, $A_1,...,A_n \in R_1$ and $A_{n+1},...,A_k \in R_2$ produces a set with attributes $A_1,...,A_k$.
If the attributes of the operands have the same names we distinguish between them  by denoting their original relation: $R_1.AttributeName$, $R_2.AttributeName$. \cite{DBSBook}
For instance, we can concatenate the information in the $book\_owner$ and $owns$  relations.

$book\_owner \times owns = r$ will result a relation with the structure as seen in equation \ref{eq:cart_struct}.
\begin{equation}\label{eq:cart_struct}
    r(owns.owner\_id, owns.ISBN,book\_owner.name,book\_owner.owner\_id, author\_name)
\end{equation}
\textbf{Join}\\
The join operator defines a relation consisting of tuples from the Cartesian product of the operand, containing only tuples satisfying the given predicate.
It is defined as $\sigma_{\Theta} (R_1 \times R_2)$ \cite{DBSBook}.

Thus, we can express the relation describing the name of all $book\_owner$ entities, that owns the book with $ISBN$ "987-1-...50-4" as both equation \ref{eq:join_s} and equation \ref{eq:join_j}.
\begin{equation}\label{eq:join_s}
    \pi_{book\_owner.name} (\sigma_{owns.ISBN = book.ISBN \wedge owns.ISBN = 987-1-...50-4}  (book\_owner \times owns))
\end{equation}
\begin{equation}\label{eq:join_j}
    \pi_{book\_owner.name} (\sigma_{book.ISBN = 987-1-...50-4} (book \Join_{owns.ISBN = book.ISBN} owns))
\end{equation}\\


Other variations of the join operation exists\cite{DBSBook}, however these will not be discussed in this report.\\
\subsubsection*{Aggregate functions}
In conjunction with the basic relational operators Aggregate functions can be used over a set of values.
Aggregate functions can take a set of values as input and get a single value as output\cite{DBSBook}. Some common aggregate functions are 
\begin{itemize} \label{aggregateFunctions}
    \item \texttt{count} --- returns the number of tuples in the set.
    \item \texttt{sum} --- returns the sum of the values in the set.
    \item \texttt{min} --- returns the smallest value in the set.
    \item \texttt{max} --- returns the largest value in the set.
    \item \texttt{avg} --- returns the average value in the set.
\end{itemize}


As we have now established both how to model database relations as well fundamental operations used to extract data (tuples) from these relations, we can now look at how to implement the relations in a database.

\subsubsection{SQL}\label{sec:SQL}
SQL is one of the most commonly used data definition language (DDL) and data manipulation languages (DML) \cite{DBSBook}.
SQL is a declarative query language, where one executes a query to instruct the database system to perform a set of operations to compute a desired result.
Collectively, the resulting set of operations is known as a transaction.
A transaction can be considered as a set of operations that is performed on data.
To a user or connected system, the collection of operations must seem a single, indivisible operation\cite{DBSBook}.
We will describe the effect of transactions in section \ref{sec:ACID}.


In PostgreSQL, all queries, both to define and to manipulate data, are executed as transactions\cite{postgres_transactions}.
As a DDL, SQL provides queries to define and modify relation schemas, as well as provide constraints for the attributes of the schema, such that data in the relation satisfies the constraint.  

\begin{lstlisting}[
    label=lst:CreatingTablesInSQL,
    language=SQL,
    caption=Implementing the relations from in equation \ref{eq:bookOwnerExample} and \ref{eq:relational_schema},
    showspaces=false,
    basicstyle=\ttfamily,
    numbers=left,
    numberstyle=\tiny,
    commentstyle=\color{gray},
    escapechar=|
 ]
    CREATE TABLE book (
        author_name varchar(250) not null,
        title varchar(250) NOT NULL,
        number_of_pages INTEGER NOT NULL CHECK(number_of_pages > 0),
        ISBN char(10) UNIQUE NOT NULL,
        primary key (ISBN)
    );
    CREATE SEQUENCE ownerSequence INCREMENT BY 1 START 1; |\label{line:sequence}|
    CREATE TABLE book_owner (
        name varchar(250),
        owner_id integer NOT NULL DEFAULT nextval('ownerSequence')
    );
    CREATE TABLE owns (
        owner_id INTEGER NOT NULL REFERENCES book_owner(owner_id),
        isbn char(10) NOT null REFERENCES book(isbn),
        PRIMARY KEY (isbn, owner_id)
    );
\end{lstlisting}

Code snippet \ref{lst:CreatingTablesInSQL} shows a possible definition of the relations from equation \ref{eq:bookOwnerExample} and \ref{eq:relational_schema}.
On line \ref{line:sequence} a sequence describing the \texttt{owner\_id} primary key for the \texttt{owner} relation has been defined.
Whenever a new tuple is inserted into the table, the sequence will generate a unique owner\_id for the data.
As a DML, SQL provides functionality to fetch data stored in the database. 
Many of the operators presented in the previous section is predefined in SQL and describe the exact same operation. 
Code snippet \ref{lst:join} shows a simple join operation, returning all book titles of books that are owned by the book\_owner with owner\_id 10.

\begin{lstlisting}[
    label=lst:join,
    language=SQL,
    caption=Query returning the title of all books owned by book\_owner with owner\_id = 10,
    showspaces=false,
    basicstyle=\ttfamily,
    numbers=left,
    numberstyle=\tiny,
    commentstyle=\color{gray},
    escapechar=|
 ]
 SELECT title FROM
 owns JOIN book ON owns.isbn = book.isbn WHERE owns.owner_id = 10  
\end{lstlisting}

In relational algebra, an equivalent operation could be described as \\$\pi_{title}\sigma_{owns.owner\_id = 10}(owns \Join_{book.isbn = book.isbn} book )$.

% Having established how one can define a database schemas using SQL as a DDL, as well as query data from the database using SQL as a DML.
\subsubsection*{ACID principles}\label{sec:ACID}

A relational database consists of several layers, each layer abstracting over concepts in the database.
The lowest level is the physical layer, which describes how the data are stored physically.
The purpose of the relational model is to abstract over the physical layer of the database.

This abstraction is known as the logical layer and allows database administrators to manage the physical storage without directly manipulating the physical data representation.
The logical layer describes what are stored and the relationships between the data.
The highest of abstraction is the view layer, which describes only part of the database.
It exists to simplify the interaction with the system. Many views may exist for the same database.
\cite{DBSBook}
We have seen that SQL can be used to describe operations performed on this abstraction, both to define and to query data.

Compared to storing data using a regular file system, a database system provides many advantages including atomicity of operations, concurrent access to data, and lowered inconsistency and redundancy of data\cite{DBSBook}.
These advantages can be directly seen in the ACID properties that databases adhere to when performing a transaction\cite{DBSBook}.
\begin{itemize} \label{ACID}
    \item Atomicity: A transaction must either be fully completed or partial side-effects of a failed transaction must be undone.
    \item Consistency: A transaction in isolation must ensure values remain consistent after a transaction has been completed or terminated.
    \item Isolation: Transactions are unaware of other transactions being executed concurrently to avoid confusion.
    \item Durability: Changes caused by a committed transaction persist even in the event of system failures.
\end{itemize}

\subsection{The Current \knox{} Databases}
The previous group set up a PostgreSQL database on \texttt{node02} to store information about word count in articles from \texttt{Nordjyske} and \texttt{Grundfos}. We will refer to this database as the WordCount database.

Figure \ref{olddatabase} shows an ER diagram of the database as we received it from the previous group.

\begin{figure}[h]
    \centering
    \includegraphics[width=\linewidth]{Images/old_db_er_diagram.png}
    \caption{ER diagram of the relational database from last year.}
    \label{olddatabase}
\end{figure}


The previous group also deployed Apache Jena Fuseki on \texttt{node01}. They chose to use HDT format, claiming that it is faster for querying than TDP\cite{knox2020}. 
The setup is used to store the knowledge graph used in \knox{}.
We will refer to this database as the Fuseki database. 

The previous group chose to develop API endpoints for both databases using Java.
The Fuseki database endpoint could be used to store knowledge graphs, and convert the data to HDT format. The API endpoint solution for the WordCount database contained a prototype for getting the ratio of how often a word occurs in an article compared to other occurring words.
There is no API for reading this data, only for writing. Currently, the database is being accessed directly by the functionality layer. 

Overall, not much code was written, and its quality is unknown due to a lack of testing and documentation.
The Java language convention was also not followed\cite{java_convention}, making it difficult to comprehend.
It appears as though no structure was established, making it difficult to navigate the code. 
Furthermore, the Fuseki database is set up such that it must be restarted every time new data is written to it - otherwise, the data cannot be fetched by the other layers\cite{knox2020}.

Based on this, we have discussed what work to do and in what order.
After having read and understood the current code, we decided that the best approach would be to start by cleaning up the inconsistencies and structure.
We then decided to document what the code does, write tests for it and to address the unfortunate database implementation, to ensure that constantly restarting was no longer required.
Finally, we needed to decouple the other layers such that they were no longer directly connected to the database.

To solve these issues, it was decided that the best approach was to discard all current implementations. 
This decision was based on several factors such as unfamiliarity with the environment and the previously mentioned poor structure. 
Discarding the current implementation would additionally allow us to build the database application using C\#, a programming language that we are more familiar with, and which is also taught in future semesters. 
While doing so, we will follow a proper structure and write tests and documentation along the way. 
Moreover, it would make the system more accessible to future students as C\# is taught on the 3rd semester.