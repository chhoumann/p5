\section{Retrospective}
The main goal of the research sprint was to gain access to all the currently deployed databases as well 
as gaining a suitable understanding of the deployed codebase. This goal was reach during the spring 
giving all group members database access and an overview of what was developed last time. 

This iteration discovered a lot of risk in how the documentation of the codebase and database is 
currently handled making the handover overly time-consuming as well allowing for too many 
misunderstandings due to the lack of documentation. To make sure this does not happen during the next 
handover an approach with increased focus on documentation was chosen to allow for an easier 
understanding of the database setup for the other teams using it and future handovers. 

This iteration discovered a lot of possible risk in how the documentation of the codebase and database i 
currently being handled. The approach taken last time did not involve a lot of focus on documentation 
which makes the project handover overly time-consuming as well as creating the possibilities of 
misunderstandings due to the lack of proper documentation. Due to request from the product owner to make 
sure this does not happen again an approach with increased focus on documentation was chosen. This 
approach both allows the other teams to easier understand how databases are running making it easier for 
them to usage them while also providing an easier handover next time.  


