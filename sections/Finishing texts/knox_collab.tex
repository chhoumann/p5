\section{Knox Collaboration}
This semester, \knox{} attempted a Scrum-inspired process, as described in section \ref{knox_collaboration}. Knox also adopted a certain amount of committees, as described in section \ref{SHARED-committees}.

Throughout this report, we documented the \knox{} collaboration from our point of view. However, these accounts mostly reflect the impediments we faced relating to the collaborative work.

Generally, we found collaboration with other groups to be beneficial. Some impediments were faced due to miscommunication, but most of the time, we found good solutions to the problems we faced.

During the first half of this project (sprint 1, 2, and 3), we primarily worked on the Search Engine. We found miscommunication throughout the pipeline to be a major problem during this period. In sprint 1, we were informed to get up to date on the current state of \knox{}, as described in section \ref{currentState}. After doing so, we were informed that we should focus our attention on getting the Search Engine working. We were told this was the focus for all groups, however other groups were not under the same impression. We were led to believe that the database layer worked fine for the Search Engine. As such, we decided to direct our attention to RDF instead. As other groups made their parts of the Search Engine operational, we were given new requirements, as described in section \ref{ssec:newRequirements}.

As a result of this, we had to shift our attention to rewriting the WordCount database implementation. This took up the entire duration of sprint 3.

After this sprint, \knox{} entered an intermediary period where all groups were asked to give suggestions for the next development steps for their layer. These suggestions had to be approved by the \knox{} PO and the \knox{} supervisors. However, many of the groups had their suggestions approved in the middle of the next sprint. Their suggestions required implementations of new APIs and databases and research thereof. Unfortunately, we were unable to accommodate their requests, as we already had received user stories and tasks, and had planned our next sprint. This led to delays in the entire pipeline.

At this point, we began receiving user stories from the \knox{} PO, which our own PO granulated with other \knox{} groups.

From our perspective, the common denominator for these issues was a lack of direction for the \knox{} project. We believe that a common backlog and a collective sprint goal for each \knox{} sprint would improve the productivity of each \knox{} group. We believe this, as we saw an increase in productivity when we introduced similar practices.
Halfway through sprint 2, we received a clear direction from the \knox{} PO, regarding the Search Engine. Because of this, we had a clear and well-defined goal for sprint 3.
After sprint 4, we began receiving user stories, which meant that we were able to granulate, prioritize, and define tasks with other groups. As a result, we had a focused direction and a goal that contributed value to the \knox{} pipeline.
Our increased productivity is reflected in our velocity analysis described in section \ref{sec:velocityAnalysis}.


% - we think this because, as we begun getting user stories, we were able to better prioritize and plan our own sprints
% - we think this helped us because we were focused on what exactly had to happen and who we had to collaborate with
% - this is reflected in our shift from being reactive to being proactive
% - we no longer had to react to every request being given to us by other groups knocking on our door, but instead were able to properly plan, prioritize, and granularize stories with the groups we were collaborating with

% because we knew WHAT to do, we also knew what NOT to do.

% - Search engine initial problem was confusing and unclear.
% - Search engine review - PO did not appear at End-End test
% - SE - did not get MVP specs before far into sprint 2 - as seen in \ref{ssec:newRequirements}
% - after SE, we were told that we did not get any user stories from the PO, and that we should suggest directions ourselves. This occurred during an 'intermediary' period where nothing happened, besides us having to suggest directions. This was our sprint 4.
% - Many other groups had their tasks approved long after we did, and they required database setup for theirs. We had to deny their requests, as we no longer had time to work on it.
% - Then, we were told (after our PO pressuring the \knox{} PO), that we would receive a new backlog of tasks, and that we should work on them.
% - After this, all validation went through the \knox{} PO, which proved to be a much better process.