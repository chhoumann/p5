\chapter{Conclusion}
\textit{This section has been made in collaboration between all groups.} \\

\noindent This sprint concluded the work with regard to getting the search engine up and running in accordance with the MVP. To see if the MVP was met, an end-to-end test of the entire system was planned for October 27th, where each group would send a representative. Each representative had the responsibility of testing their part of the pipeline with real data. The end-to-end test did not go as planned. The preprocessing layer of the pipeline had trouble communicating with the knowledge layer due to the delay in launching a communication API. Furthermore, the JSON schema for \textit{Nordjyske Mediehus} articles could not be verified by the knowledge layer.
The lower layers also had trouble communicating with each other, due to routing problems, however, this was fixed at the first end-to-end test. By the end of the test, the \textit{Grundfos} data did manage to enter the database. The search result would appear on the UI which could redirect the user to the associated PDF. In the beginning, the UI send the wrong flags to search for in the database, which was fixed during the end-to-end test.\\

\noindent As of October 29th the MVP has not been met. This means sprint three has been extended to the 3rd of November, where a second end-to-end test took place. \\
.......
RETTES HER EFTER 3rd NOVEMBER!
.......
\subsection{Scrum-of-Scrums Retrospective}
The retrospective is used for evaluating sprints.....
Noget med hvad det bruges til + kilde

\subsubsection{What Went Well}
The collaboration between Scrum teams improved as time went by. The fact that the groups were physically located close to each other made the communication between groups easy. People also appreciated the planned end-to-end test to identify flaws in the current pipeline and fix them in due time.

\subsubsection{What Did Not Go Well}
The first issue during this part was that the presentation of the whole Knox project was not well-defined. This resulted in misunderstandings regarding the goals of the entire Knox project. This meant that the Scrum teams had a hard time, in the beginning, defining \textit{sprint backlog} tasks. From a student's perspective, it also seemed like the supervisors were unclear about the scope of the project. This uncertainty, lack of \textit{backlog} tasks, and misunderstandings led to a summit, which resulted in a new goal defined by the minimum viable product (MVP).\\

\noindent The \textit{product owner} (PO) attended the Scrum-of-Scrums meetings to help clarify confusions. In hindsight, we did not use the PO enough and at times had a hard time prioritizing our tasks. A reason for this could be that groups were so dependent on each other and new tasks kept occurring. \\

\noindent Lastly, even though communication was one of the things that went well for the sprints, it still has to be improved. Miscommunication kept happening in regard to defining tasks and deciding who was responsible for the tasks between the layers, as well as the committees lacking purpose and feeling like they wasted time on their meetings.

\subsubsection{Actions for Future Sprints}
Goals for the next sprints were better sprint planning and making sure that everybody has a clear understanding of the product goal. 
To help unravel confusions regarding tasks we have to be better at making contact with the PO especially if new tasks arise during a sprint. 
To facilitate communication between the Scrum teams, it was decided that the groups should try to use user stories between the groups. Meaning that, when new tasks are given to a Scrum team, a story on how it should work should be included.
Lastly, the Scrum teams have to be better at communicating whether or not they are on time to reach the sprint goal. If they are not, they should be better at asking for help from other groups. 