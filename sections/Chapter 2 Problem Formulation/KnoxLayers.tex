\section{The Layers of the \knox{} Pipeline}
The Knowledge Engineering Toolkit or Knox is a project which goal is to provide a flexible toolkit whose components can be combined 
to solve large scale problems. The long-term vision for Knox is to provide access to chatbots and personal assistants that can  
provide a proven answer to the users' questions through the usage of knowledge extraction, natural language processing, fact- 
checking, explainable AI.
The overall goal for the 2nd iteration of Knox is to complete a search engine with the provided components from the 1st iteration. 
Furthermore, a shared interface for the front-end and back-end and an advancement of its components is expected. 


The Knox project itself is currently split into a 4-layered pipeline structure. The first layers of the pipeline is the Pre-
processing layer which takes image files as input. Here the goal is to convert the image files into text files using optical image 
recognition. 


Once the first layer has completed its objective the files are then sent to the next stage of the pipeline being the Knowledge 
layer. The main task here is to extract as much data as possible from the files generated by the previous layer using natural 
language processing. Here the data is extracted in the form of RDF triples which consists of a subject, an object, and the predicate 
connecting them. 


Next up the 3rd layer also known as the data layer handles the data management and integration across the board. This layer is 
responsible for managing all the data extracted during the different stages of the pipeline and making sure it is available for the 
next stages when needed. 


The final stage of the pipeline is the functionality layer. This layer handles all the functionalities that are available for the 
end-user. These functionalities cover fact-checking and providing an explanation for the given prediction.


With an overview of the current state of the overall main objectives of the project  established a more in-depth look at the current 
state of the data layer is now needed to continue the development of the layer. 

