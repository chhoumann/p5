\section{Semester Context}
This semester is thematically based on the students learning how to collaborate as teams, between teams.

To facilitate learning, students work in teams on parts of a complex project. 
Each team has a specific role, and will also, to some degree, provide support to teams that they are collaborating with.
To this, students are taught certain software process models, and are expected to use them in their work.

This is supposed to simulate the environment of an actual software project.

, is part of the Knox multi-project.
We will describe Knox in more detail in the next chapter.
However, it is important to note that Knox is developed using Agile software development practices.
This is also the process model that we will be using in this semester.

The choice to use Agile software development practices is not a technical decision, but rather a choice of a process model that is more suitable for the type of work that we are doing.
We find the benefits of an Agile approach superior to other models, considering the scope, details, and requirements for the project.
To specify, we will be dealing with a lot of unknowns, which means that the flexibility of an agile approach will be almost crucial.

A case could be made for the waterfall model, in that we have a certain deadline for delivering a report, and the phases of writing the report (and the deadline) are known beforehand. 
However, the contents of the report is what's important, and that will only become better by 1. following and learning from the designated process, and 2. doing good work, which the agile approach — in our estimation — would help us do.

Similarly, a case could be made for the integration and configuration model.
We could do much of our assigned task by integrating and configuring various 3rd party components and systems. 
However, we would like to learn as much as we can. 
As such, the approach of using the work of others is not ideal — even if you learn a few things by implementing it.