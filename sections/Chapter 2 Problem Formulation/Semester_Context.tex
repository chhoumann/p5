\section{Semester Context}
This semester is thematically based on students learning how to collaborate as teams on a complex project. 
To accommodate this, students work in interdependent teams on separate parts of a project. 
Common methods are used inbetween each team as described in Section \ref{knox_collaboration}. 

Each team has a specific role, and should, to some degree, provide support to teams that they are collaborating with.
To do this, students are taught the iterative-incremental software process model \cite{SoftwareProcessModels}, and are expected to use it in their work. The teams have common means of collaboration (see Section \ref{communication_in_knox})
This is supposed to emulate the environment of an actual software project. 


% However, it is important to note that Knox is developed using Agile software development practices.
% This is also the process model that we will be using in this semester.
% 
% The choice to use Agile software development practices is not a technical decision, but rather a choice of a process model that is suitable for the type of work that we are doing.
% We find the benefits of an Agile approach superior to other models, considering the scope, details, and requirements for the project.
% To specify, we will be dealing with many of unknowns, which means that the flexibility of an agile approach will be beneficial.



% A case could be made for the waterfall model, as we have a known deadline for delivering a report, in which the phases of writing are known beforehand. 
% However, the contents of the report is what is important, and that will only become better by 1. following and learning from the designated process, and 2. doing good work, which the agile approach should help us do.
% The work that we will do was, when we started the project, largely unknown.
% 
% Similarly, a case could be made for the integration and configuration model.
% We could do much of our assigned task by integrating and configuring various 3rd party components and systems. 
% However, we would like to learn as much as we can. 
% As such, the approach of using the work of others is not ideal — even if you learn a few things by implementing it.
