\section{Committees}\label{SHARED-committees}
\textit{The following section has been written in collaboration with other Knox project groups.}

Committees were established, as an attempt to ensure cooperation between the different groups. Several committees were founded.

\subsection{Scrum-of-Scrums}
Scrum-of-Scrums was introduced in the Knox project as a means of helping teams obtain transparency as well as the possibility to adapt to a common development process and at the same time be informed of the progress of other groups\ \cite{agile}.
The meetings were held on a weekly basis and helped ensure that the groups had a common discourse during each sprint. 
The PO was also present during meetings, in order to ensure that the decisions made would be meaningful elements in Knox. 
During the first three sprints, the main topic of the meetings was the development of an minimum viable product.
The Scrum-of-Scrums meetings helped create the above-mentioned qualities.

Scrum-of-Scrums was beneficial to the communication between teams and helped reach agreements on how to ensure that the development process was carried out in a manner that focused on cooperation between the teams.

\subsection{DevOps}
DevOps combines various practices and tools and emphasizes the importance of implementing a DevOps culture. According to a survey, 99\% of the respondents believed that using DevOps had impacted their organization in a positive way \cite{Atlassian}. Using DevOps helps improve the speed of the work process, along with communication and collaboration between teams.

The DevOps committee focused on streamlining the development process, and decisions on the use of Continuous Integration were discussed during these meetings. 
Important decisions such as “definition of done” along with requirements for code standards and the rules for server use were also discussed during these meetings.
Several of the members of the committee had zero to little experience with DevOps when starting out, hence the committee worked as a platform of information sharing and collaborating on finding the most viable solutions. This helped reach a common understanding of DevOps, as well as implementing methods such as definition of done, to ensure that the process was streamlined.

\subsection{Writing committee}
During Scrum-of-Scrums, it was agreed upon that each team would collaborate on parts of the written project. The meetings were held in order to collaborate on writing these parts, to correct the written material, and ensure that it lived up to collective standards of writing.


% Indsæt UI

\subsection{Committees effect on the sprint planning}
\textit{This section is not part of the collective writing.}
One of the main goals of establishing committees that surveyed most of the \knox{} project was to get ahead of potential problems that could arise with multiple teams working on the same project. 
This was done by discussing decisions that would have a large effect on the development teams to these committees and allowing them to make decisions with the input from all other  \knox{} teams as well as the Product Owner of \knox{}. 
The wanted outcome was to make the planning phase easier for the groups, by giving them knowledge of what other teams expected to develop and release, as well as which components the various teams depended on.

A downside to this approach was that it meant the development teams could not actively take large decisions themselves, and instead often had to wait on an official decision from the committees. 
Another example was that one of the primary goals being to create a UI component. 
This task was locked behind the UI committee taking a decisions on how to move forward with the UI. 
Teams could not start working on their component in the old setup due to the UI committee not initially knowing if it wanted to keep the old UI or wanted to change framework and build a new one up from scratch.


