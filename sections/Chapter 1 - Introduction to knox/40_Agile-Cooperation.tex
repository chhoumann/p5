\section{Agile cooperation}

\subsection{Our Agile Approach}
In the previous iteration of Knox, Scrum was used as a framework for planning in and between the project groups. It has therefore been chosen to continue to use a Scrum-inspired approach for the project. The reason why it was Scrum-inspired and not only Scrum was because many groups had not used Scrum before, and therefore some aspects were not fully realized. For example, for the first couple of sprints, there were no sprint reviews or retrospectives in the Scrum-of-Scrum meetings.  

Deviations were made from the standard Scrum to manage overhead in the Scrum-of-Scrums work environment. Multiple committees (described in \autoref{SHARED-committees}) were implemented to reduce the duration of Scrum-of-Scrums meetings and the responsibilities of Scrum masters. The goal of the committees was to assign responsibility for tasks affecting all groups.
This led to a lot of collaboration between the groups, which ensured communication and transparency. 

\subsection{Roles and Events}
The theory and definitions of the different Scrum terminology will not be discussed in this section. All definitions are taken from The Scrum Guide by Ken Schwaber and Jeff Sutherland
\footnote{The Scrum Guide (2020) - \url{https://Scrumguides.org/docs/Scrumguide/v2020/2020-Scrum-Guide-US}}.


The Knox project consists of eight Scrum teams, who work on different features of the pipeline. In general, a Scrum team has their own Scrum master and product owner. The “global” project owner for the whole Knox project was an older student who had worked on the previous iteration of Knox. The product owner also worked closely with the group supervisors during the project.

All groups made a backlog for the project together, however, all tasks were somewhat loosely defined, and it was up to the assigned groups to specify them. Sprint planning was done during the Scrum-of-Scrum meetings, which were held weekly throughout the project. Sprint reviews began during the third sprint and were made with a PowerPoint containing the finished and in-progress tasks of the groups. A component diagram of the Knox project was also updated each week. Sprint retrospective was held by the Scrum masters of all groups between sprints.
