\section{Agile cooperation}\label{knox_collaboration}
\textit{The following section has been written in collaboration with other Knox project groups.}


\subsection{Agile approach in Knox}\label{common_agile_methods}
In the previous iteration of Knox, Scrum was used as a framework for planning in and between the project groups. It has therefore been chosen to continue to use a Scrum-inspired approach for the project. The reason why it was Scrum-inspired and not only Scrum was because many groups had not used Scrum before, and therefore some aspects were not fully realized. For example, for the first couple of sprints, there were no sprint reviews or retrospectives in the Scrum-of-Scrum meetings.  

Deviations were made from the standard Scrum to manage overhead in the Scrum-of-Scrums work environment. Multiple committees (described in \autoref{SHARED-committees}) were implemented to reduce the duration of Scrum-of-Scrums meetings and the responsibilities of Scrum masters. The goal of the committees was to assign responsibility for tasks affecting all groups.
This led to a lot of collaboration between the groups, which ensured communication and transparency. 

\subsection{Roles and Events}\label{roles_and_events}
The theory and definitions of the different Scrum terminology will not be discussed in this section. All definitions are taken from The Scrum Guide by Ken Schwaber and Jeff Sutherland
\footnote{The Scrum Guide (2020) - \url{https://Scrumguides.org/docs/Scrumguide/v2020/2020-Scrum-Guide-US}}.


The Knox project consists of eight Scrum teams, who work on different features of the pipeline. In general, a Scrum team has their own Scrum master and product owner. The “global” project owner for the whole Knox project was an older student who had worked on the previous iteration of Knox. The product owner also worked closely with the group supervisors during the project.

All groups made a backlog for the project together, however, all tasks were somewhat loosely defined, and it was up to the assigned groups to specify them. Sprint planning was done during the Scrum-of-Scrum meetings, which were held weekly throughout the project. Sprint reviews began during the third sprint and were made with a PowerPoint containing the finished and in-progress tasks of the groups. A component diagram of the Knox project was also updated each week. Sprint retrospective was held by the Scrum masters of all groups between sprints.

\subsection{Our Agile Approach}\label{our_agile_approach}
\textit{This section is not part of the collective writing.}

In our group, we have strived to utilize as much of Scrum as we could.
As is the case in Scrum, a given group has one product owner, one Scrum master, and some developers.

The product owner is, as the name implies, responsible for the product. In practice, this makes them accountable for maximizing the value of the product. To this end, the product owner manages the product backlog, and is responsible for the product vision. As we participate in a Scrum-of-Scrum hierarchy, the product owner is responsible for reporting progress on the project to the Knox product owner as well.

The Scrum master is responsible for enforcing Scrum. More importantly, the Scrum master is responsible for guiding the team towards continuous improvement, which leads to increased velocity and happiness.

Increasing velocity is especially desirable, as it indicates that the team is increasing their productivity. Velocity is measured by taking the sum of the estimates of each completed task in a sprint\cite{sutherlandScrumArtDoing2014}. 

Our general approach imitates the Scrum method. First, we have a product backlog filled with potential improvements to the project. Each sprint, we, the scrum team, take the most important of these and plan an increment based on them. We work towards this increment during the sprint. Each day, we have a short Daily Scrum, where we discuss progress and eventual obstacles. At the end of the sprint, we have a Sprint Review, where we discuss the progress of the sprint and should happen next. Finally, we have a Sprint Retrospective, where we discuss how the sprint went and what we learned.

To keep consistent with the progression of the project, our report sections will follow the flow of our Scrum workflow.