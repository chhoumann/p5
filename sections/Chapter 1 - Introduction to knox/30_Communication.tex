\section{Communication}\label{communication_in_knox}
\texit{The following section have been written in collaboration with other Knox project groups.}


To collaborate on the project, several communication tools were used, one being ClickUp \cite{clickup}. ClickUp was primarily used for time and task management of the different sprints for each group. In ClickUp, it is possible to create different boards and assign tasks to each board. The boards can be shared between different groups, which gives the opportunity to have a shared backlog as well as a backlog for each individual group. Hereby, the development process becomes more transparent for each group, as well as the product owner. It is also possible to mark tasks with a workload, so it is possible to see how time-consuming a task is estimated to be. ClickUp is thus an implicit communication tool, making it possible to see how far each group is. If one group has very few important, time-consuming tasks, they should be open to helping other groups, if they have an important task, which they are not able to fulfill within the sprint.
\\\\
Because of the layer structure in the project workflow, many small meetings took place during each sprint. Therefore, it was decided that a post-it note should be taped to each group room door. The note should say, which groups were located in the room, as well as which layer and project they were working on, and their typical meeting hours. 
Many of these small meetings were arranged through Slack \cite{slack}. Slack was used for all direct communication between the different groups, as well as scheduling of meetings. It was also used for knowledge sharing, such as information about which servers to use. Slack has features such as channels, threads, and direct messages, and many of these features were frequently used. Each group was given their own channel, where people of other groups could write information or questions to the members of the channel. During the sprints, several committees were formed, and each committee was also given a channel in Slack. The committees were formed as a communication path between the groups and layers. The committees were specific forums for concrete problems, concerning collaboration on the project in general. The committees will be described in more detail in \ref{SHARED-committees}.
