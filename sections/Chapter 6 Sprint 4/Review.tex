\section{Review}
The goal for this sprint was primarily to catch up on documenting progress and previous sprints. As a side-goal, we needed to gather information about our next sprints now that the Knox goal of implementing the search engine has concluded.

\subsubsection*{What has been accomplished?}
We managed to catch up on writing the documentation for the previous sprints. During our internal review, we found many areas which we are looking to improve in future sprints. These areas were discovered by performing a Premortem review of the documentation, which we found to be a great way to get a sense of what we were missing. A Premortem is a method in which you imagine failure before you actually deliver the product, and you try to learn from those failures\cite{PremortemSadanForbedrer}.

We asked the Knox product owner for user stories, and as a result, we received three user stories which we can work on in the next sprints.

\subsubsection*{What has changed in the environment?}
Due to feedback, meetings have been shortened in duration and reduced in scope. 

As mentioned, we now have user stories which we can work on in the next sprints. This means we now have a clearer direction for the remainder of the sprints.

\subsubsection*{What next?}
The user stories we received from the Knox product owner are included below. The first two are directed at us, and the last one is for all \knox{} layers.

\userStory{the knowledge layer}{to be able to insert Triples into a storage}{can access it later}
\userStory{the functionality layer}{to be able to query an RDF graph}{can use it in search results, fact checking and virtual assistant applications without having to worry about changes to the database}
\userStory{an administrator}{to see the various progress, processes, statistics, and more on a website}{can get an overview of the pipeline}

At this time, we also know that there are other issues that we need to address. To this, we will need to add more user stories for our internal group.

\begin{itemize}
    \item Granulate product owner's user stories with the other \knox groups to increase the accuracy of the sprint planning.
    \item Prepare for tasks related to implementing a RDF database, which the user stories from the Knox product owner imply a need for.
    \item Prepare for UI related tasks.
\end{itemize}

These tasks will be processed accordingly during sprint planning in the next sprint.