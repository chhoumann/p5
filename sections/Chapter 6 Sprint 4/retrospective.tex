\section{Retrospective}
We have received three user stories from the PO, and can use them to plan the next sprint.  We have written 20+ pages in the report. These have been reviewed by team members.  We have thus accomplished the sprint goals. 

We are waiting for validation from the project supervisor (report review).

\subsection{Sprint goal}
The goal of this sprint was to gather information about what to do next sprint, write missing report sections from the previous sprints and revise the report based on the feedback received from the supervisor. 

\subsection{Reflection on the sprint}
During previous sprints, focus was dedicated on completing the implementation and deployment of the wordcount database and WordCount API. This resulted in parts of the report being unfinished, missing or in need of revision. Consequently, this meant that our focus this sprint had to be directed towards catching up on the report.
During the third sprint, a major contributing factor causing problems was identified: the length of the sprint. Previous sprints had been long - a total of four weeks - and resulted in a lack of overview and lack of flexibility. To accommodate this, we decided to reduce the sprint length significantly to two weeks. The benefit of this approach became evident as we were now able to better grasp the requirements of the sprint goal and appropriately granulated tasks to fulfill it. As the sprint was completed, the resulting improvements could be observed clearly in the number of tasks completed and the number of story points that we achieved. 
Another change was the addition of user stories to better define our tasks as it improved our understanding of the requirements within the sprint. This also made it easier for the product owner of the group to create tasks without having to consider specific implementation details and also resulted in less duplicate tasks, which had been an issue previously.
A final change that improved our focus and workflow during the sprint was the decision to postpone addressing issues for other groups until it suits us best. During the previous sprints, we would immediately shift our attention other groups' issues when they were brought up, which negatively impacted our plan and resulted in a loss of focus.

\subsection{What we learned}
We learned that it is easier to keep an overview in smaller scale sprints as the resulting granulation of tasks is easier to do. When issues or new tasks were identified by our PO in previous sprints, we had more plans to change to accomodate new demands. With the smaller sprints changes to the plan are easier to handle. 
We also learned that, during previous sprints, a lot of time had been spent creating tasks that were unnecessarily specific, and the details of the implementation were often unclear at the time of creating them. We found that creating user stories and defining the tasks based on these made the process easier. It also allowed the individual group member to have the freedom to decide how to fulfill the user story which resulted in a smoother workflow.  
Focusing on finishing the current tasks within our group before addressing issues reported to us by other groups improved our efficiency as we no longer needed to switch context frequently.

\subsection{How we can improve in the next sprint}
\subsubsection{Start doing}
\begin{itemize}
    \item User stories with other groups
    \item Enforce report writing after anything relevant has been done
\end{itemize}
\subsubsection{Continue doing}
\begin{itemize}
    \item User stories
    \item Premortem
\end{itemize}

Having concluded the sprint, we will proceed to the next sprint.