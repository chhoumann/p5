\section{Sprint Planning}
\subsection{Knox Goal}
\subsection{User stories}
The \knox{} product owner have given us three user stories, which we can use to detail the coming tasks for the sprint.
Our own product owner will split these user stories, which we consider to be \textit{epics}, together with the involving layer \cite{Epics}.
This will help us understand how to accommodate the needs of the involving layer and prioritize the stories.\cite{UserStories}
This in turn will help us generate as much value to the pipeline as quickly as possible.

\userStory{the knowledge layer}{insert Triples into a storage}{others can access it later}
\userStory{the functionality layer}{query an RDF graph}{can use it in search results, fact checking and virtual assistant applications}
\userStory{functionality layer}{query an RDF graph in a standardized way}{do not have to worry about changes to the database}

\subsubsection*{Granulating the epics}
After talking to the functionality layer, we found out details about how they want database access to work. 
\begin{enumerate}
    \item They want to be able to send SPARQL queries directly to an endpoint to fetch the data.
    \item They would prefer if the response was in XML format. They have given us an example.
    \item They do not know the size of the data they will be querying.
    \item The route for GET request should have a string parameter called "query" containing the SPARQL query as a string.
    The query will be in URL-coded format.
\end{enumerate}



\todo{reference to what an epic is}

\textbf{Sprint goal}


\textbf{Backlog \& Increments}

%The sprint goal for this group
%How does our sprint goal help with the overall sprint goal
%The purpose and priority for each increment
%Show the available backlog
